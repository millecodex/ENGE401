%---------------------------------------------------
% Differentiation
%---------------------------------------------------
\chapter[Differentiation]{Differentiation}

\section{Rate of Change}
We are all familiar with the concept of average speed. If you travel a distance of 120 km in 2 hours then your average speed is 60 kph. There is a formula that you will have met 

\begin{center}
Average speed $ =\frac{\text{distance travelled}}{\text{time elapsed}}$
\end{center}\par
A distance/time graph can be drawn. The average speed can be expressed using function notation 

\begin{center}
Average speed $ =\frac{s (b) -s (a)}{b -a}$
\end{center}\par
Finding the average rate of change is important in many contexts and in fact the average rate of change can be defined for any function. 

The average rate of change of the function $y =f (x)$ is $\frac{\text{change in }y}{\text{change in }x}$ or $\frac{f (b) -f (a)}{b -a}$\hfill(1) 

The average rate of change is the slope of the \textbf{secant line} between $x =a$ and $x =b$ on the graph of $f$,\ that is the slope of the line that passes through $(a ,f (a))$ and $(b ,f (b))$. 

\textbf{Example} Calculate the average rate of change for the function $f (x) =x^{2} +4$ between the following points:
\begin{tasks}[counter-format=(tsk[1]),column-sep=3em](3)
\task $x =2$ and $x =6$ \\
\textbf{Solution}\\ Using the function notation in (1) above, \[\frac{f(2)-f(6)}{2-6}=\frac{8-40}{-4}\]
\[=\frac{-32}{-4}=8\]

\task $x =5$ and $x =10$ \\
\textbf{Solution}\\
\[\frac{f(5)-f(10)}{5-10}=\frac{29-104}{-5}=15\]

\task $x =a$ and $x =a +h$\\ ($h \neq 0$) \\
\textbf{Solution}\\
\[\frac{f(a)-f(a+h)}{a-(a+h)}\]
\[=\frac{a^2+4-([a+h]^2-4)}{a+h}\]
Can this be simplified further?
\end{tasks}

%---------------------------------------------------
% 
%---------------------------------------------------
\subsection*{Tangents}
We now investigate the process of changing the value of $(b -a)$ in formula (1). As $(b -a)$ is made smaller and smaller the slope of the secant approaches the slope of the tangent at $x =a$. The notation for this process is as follows. 

\textbf{Definition:} The tangent line to the curve $y =f (x)$ at the point $P (a ,f (a))$ is the line through $P$ with slope $m =\Lim{x\to a}\frac{f (x) -f (a)}{x -a}$ provided that the \textit{limit} exists. This means that as the value of $x$ gets close to $a$ the function remains smooth. Imagine zooming in on a function, from far away it may appear smooth, but up close it could have some gaps or discontinuities.  Limits do not exist at sharp transitions in a graph, or where the function does not exist (think of piecewise functions). 

We sometimes refer to the slope of the tangent line to a curve at a point as the slope of the curve at that point. The idea is that if we zoom in far enough towards the point then the curve looks almost like a straight line. The more we zoom in the more the parabola looks like a straight line. 

Using function notation for the tangent line is usually easier to use and is often preferred. The slope of the secant line between $x =a$ and $x =a +h$ is $\displaystyle \frac{f (a +h) -f (a)}{h}$. This looks familiar from \textsc{example} 3 above. We can now use this as a definition:

\begin{tcolorbox}
$$\text{slope }=m=\Lim{h\to 0} \frac{f(a+h)-f(a)}{h}$$
\end{tcolorbox}
This is limit notation, $\Lim{h\to 0} $, and we would say `the limit as $h$ approaches $0$'. Note that if $h=0$ the function is now undefined (math error). So $h$ is allowed to get close to zero, but not actually equal zero.

\subsection*{Velocities}
When this process is applied to the average velocity the result of computing the average velocity over shorter and shorter time intervals produces the \textbf{instantaneous velocity}.\ Let $s =f (t)$ be the displacement at time $t$ then the \textbf{instantaneous velocity} is 
$$v (a) =\Lim{h\to 0}\frac{f (a +h) -f (a)}{h}$$
This is often referred to as the \textit{velocity at} $t=a$. 

Suppose $y$ is a quantity that depend on another quantity $x$. Thus $y$ is a function of $x$ and we write $y =f (x)$. If we change from $x_{1}$ to $x_{2}$ then the change in $x$ is called an increment in $x$ and is denoted by $ \Delta x =x_{2} -x_{1}$ and the corresponding increment in y is denoted by $ \Delta y =y_{2} -y_{1}$. 

The average rate of change of $y$ with respect to $x$ over the interval $\left [x_{1} ,x_{2}\right ]$ is denoted by 
$$\frac{ \Delta y}{ \Delta x} =\frac{f (x_{2}) -f (x_{1})}{x_{2} -x_{1}}$$
and can be interpreted geometrically as the slope of the secant line. As we did with velocities we can compute the average rate of change over smaller and smaller intervals by letting $x_{2}$ approach $x_{1}$ and therefore letting $ \Delta x$ approach $0$. The limit of this process is called the instantaneous rate of change of $y$ with respect to $x$ at $x =x_{1}$. This is interpreted geometrically as the slope of the tangent to the curve $y =f (x)$ at $(x_{1} ,f (x_{1}))$. 

\begin{tcolorbox}
$$\text{Instantaneous rate of change }=\Lim{\Delta x \to 0} \frac{ \Delta y}{ \Delta x} =\Lim{x_2 \to x_1}\frac{f (x_{2}) -f (x_{1})}{x_{2} -x_{1}}$$
\end{tcolorbox}

%---------------------------------------------------
% Derivatives from 1st Principles
%---------------------------------------------------
\section{Derivatives from 1st Principles}
Because the expression $\Lim{h \to 0}\frac{f (a +h) -f (a)}{h}$ occurs so widely it is given a special name and notation. 

The derivative of a function at a number $a$, denoted by $f^{ \prime } (a)$ is 
\begin{tcolorbox}
Definition of the derivative:\qquad$\displaystyle f^{ \prime } (a) =\Lim{h \to 0}\frac{f (a +h) -f (a)}{h}$
\end{tcolorbox}

The process of finding the derivative using the above definition is called finding the derivative \textit{from first principles}. So far we have found a derivative of a function $f$ at a fixed number $a$. If we replace $a$ in this equation with a variable $x$ we obtain 
\[f^{ \prime } (x) =\Lim{h \to 0}\frac{f (x +h) -f (x)}{h}\]
given any number $x$ for which this limit exists. We assign to $x$ the number $f^{ \prime } \left (x\right )$. So we can regard $f^{ \prime }$ as a new function which we call the derived function or the derivative of $x$. 

\subsection*{Alternate Notation for the Derived Function}
Given the notation $y =f (x)$ for the function the following alternative notations for $f^{ \prime } \left (x\right )$ are common: 
$$f^{ \prime } (x) =y^{ \prime } =\frac{d y}{d x} =\frac{d f}{d x} =\frac{d}{d x} f (x) \text{, also } =D f (x) =D_{x} f (x)$$
The symbols $D$ and $\frac{d}{d x}$ are called \textit{differential operators.} 


\subsection*{Derivatives of Polynomial Functions}
The constant function $f (x) =c$ is considered a polynomial of degree zero. Using the method of first principles we can find the derivative as follows: 
$$f^{ \prime } (x) =\Lim{h\to 0}\frac{f (x +h) -f (x)}{h} =\Lim{h\to 0}\frac{c -c}{h} =\Lim{h\to 0}0 =0$$



\subsection*{Higher Power Polynomials}
When $f (x) =x$ it can be shown from first principles that $f^{ \prime } (x) =1$. Similarly when $f (x) =x^{2\text{}}$it can be shown that $f^{ \prime } (x) =2 x$ and when $f (x) =x^{3}$ it can be shown that $f^{ \prime } (x) =3 x^{2}$. 

As an example let us prove the formula for $f (x) =x^{4}$ from first principles
\begin{align*}f^{ \prime } (x) &  = \Lim{h\to 0}\frac{f (x +h) -f (x)}{h} \\
 &  = \Lim{h\to 0}\frac{(x +h)^{4} -x^{4}}{h} \\
 &  = \Lim{h\to 0}\frac{x^{4} +4 x^{3} h +6 x^{2} h^{2} +4 x h^{3} +h^{4} -x^{4}}{h} \\
 &  = \Lim{h\to 0}\frac{4 x^{3} h +6 x^{2} h^{2} +4 x h^{3} +h^{4}}{h} \\
 &  = \Lim{h\to 0}(4 x^{3} +6 x^{2} h +4 x h^{2} +h^{3})\text{ here, substitute }h=0 \\
 f'(x)&  = 4 x^{3}\end{align*}

This pattern will follow for any similar polynomial: If $f (x) =x^{n}$ then $f^{ \prime } (x) =n x^{n -1}$. Or alternatively 
\begin{tcolorbox}
The Power Rule for Differentiation:\qquad$\displaystyle \frac{d}{d x} (x^{n}) =n x^{n -1}$
\end{tcolorbox}
	
This pattern implies that $n$ must be a positive integer. It can be shown that from the definition of a derivative $\frac{d}{d x} \genfrac{(}{)}{}{}{1}{x} = -\frac{1}{x^{2}}$ or $y =x^{ -1}$ then $\frac{d y}{d x} = -1 \times x^{ -2}$, which proves the power rule for $n = -1$. 

Similarly if the exponent is a fraction it can be shown that the power rule holds e.g. if $f (x) =\sqrt{x}$ then $f^{ \prime } (x) =\frac{1}{2 \sqrt{x}}$ or $f (x) =x^{\frac{1}{2}}$ then $f^{ \prime } (x) =\frac{1}{2} x^{ -\frac{1}{2}}$. It can be shown that the power rule holds for any real number $n$. 

%---------------------------------------------------
% Standard Derivatives
%---------------------------------------------------
\section{Standard Derivatives}
The basic functions have easily repeatable patterns to find their derivatives. The common ones are summarized in the table below:
\begin{center}
\begin{tabular}{ccll}	
	\toprule
	Function& Derivative&&Notes\\\midrule
	$f(x)$ & $f'(x)$  &&notation\\ \midrule
	$x^n$ & $nx^{n-1}$ &&`power rule'\\ \midrule
	$e^x$ & $e^x$  && exponential\\ \midrule
	$\ln(x)$ & $\frac{1}{x}$ &&logarithmic\\ \midrule
	$\sin(x)$ & $\cos(x)$  && \\ \cmidrule{1-2}
	$\cos(x)$ & $-\sin(x)$ && trigonmetric\\ \cmidrule{1-2}
	$\tan(x)$ & $\sec^2(x)$ && \\ \bottomrule
\end{tabular}
\end{center}


\subsection*{The Natural Exponential}
Recall the natural exponential function from section~\ref{sec:naturalExponential}. Here we can see why precisely it is so special. Using limit notation, we can say that $e$ is the number such that $\Lim{h\to 0}\frac{e^{h} -1}{h} =1$. The derivative is:

\begin{tcolorbox}
The Derivative of the Natural Exponential Function
$$\frac{d}{d x} \left (e^{x}\right ) =e^{x}$$
\end{tcolorbox}

The natural exponential function is unique becuse \textbf{it has its own derivative!} Geometrically this means that the slope of the tangent at any point is the same as the y-coordinate, $f(x)$, of that point. 

%---------------------------------------------------
% max, min and tangents
%---------------------------------------------------
\section{Maximums, Minimums, and Tangents}
\subsection*{What does $f^{ \prime }$ tell you about $f$?}
Because $f^{ \prime } (x)$ represents the slope of the curve $y =f (x)$ at the point $\left (x ,f \left (x\right )\right )$ it tells us about the direction the curve is proceeding at that
point. $f^{ \prime } (x)$ therefore gives us information about $f (x)$. $f^{ \prime } (x) >0$ means that the tangent line has a positive slope and $f^{ \prime } (x) <0$ means that the tangent line has a negative slope. 

When $f^{ \prime } (x) >0$ on an interval then $f$ is increasing on that interval 

When $f^{ \prime } (x) <0$ on an interval then $f$ is decreasing on that interval. 

When $f^{ \prime } (x) =0$ the tangent is horizontal (i.e. parallel to the $x$-axis). We call the point where this occurs a turning point.. 

When
to the left of a turning point $f^{ \prime } (x) >0$ and to the right of the turning point $f^{ \prime } (x) <0$ the turning point is called a \emph{local maximum}. 

When to the left of a turning point
$f^{ \prime } (x) <0$ and to the right of the turning point $f^{ \prime } (x) >0$ the turning point is called a \emph{local minimum}. 

When to the left of a turning point
$f^{ \prime } (x) >0$ then to the right of the turning point $f^{ \prime } (x) >0$ the turning point is called a \emph{point of inflection}. 

Finally $f^{ \prime } (x)$ could be $ <0$ to the left of the turning point and to the right also. This is also called a \emph{point
	of inflection}. 

\subsection*{What does $f^{ \prime  \prime }$ tell you about $f$?}
We know that $f^{ \prime  \prime } (x) =\left (f^{ \prime } (x)\right )^{ \prime }$ so $f^{ \prime  \prime }$ is related to $f^{ \prime }$ in the same way as $f^{ \prime }$ is related to $f$. 

When $f^{ \prime  \prime } (x) >0$ on an interval then $f^{ \prime }$ is increasing on that interval. A curve that is \emph{concave upwards}
has this characteristic. To the left of the turning point the slope is negative and approaches zero as the value
of $x$ approaches the turning point. To the right of the turning point the slope is positive
and increases as the value of $x$ increases. 

When $f^{ \prime  \prime } (x) <0$ on an interval then $f^{ \prime }$ is decreasing on that interval. A curve that is \emph{concave downwards} has this characteristic. 

Whenever
a curve changes from concave upwards to concave downwards, the point at which this takes place is called a \emph{point of inflection}.
\ Similarly whenever a curve changes from concave downwards to concave upwards, the point at which this takes
place is also called a \emph{point of inflection}. 



%---------------------------------------------------
% The Product, quotient and chain rules
%---------------------------------------------------
\section{The Product, Quotient, and Chain Rules}

%-----------------------------------------------
\subsection{The Product Rule}
Let $f (x) =x$ and $g (x) =x^{2}$. What is the derivative of $f (x) \times g (x)$? The question helps to show that the answer is NOT $f^{ \prime } (x) \times g^{ \prime } (x)$ 

$f (x) \times g (x) =x \times x^{2} =x^{3}$ and we know the derivative of $x^{3}$ is $3 x^{2}$. Also we know that $f^{ \prime } (x) =1$ and $g^{ \prime } (x) =2 x$ so $f^{ \prime } (x) \times g^{ \prime } (x) =1 \times 2 x =2 x$ not $3 x^{2}$. 

So the derivative of the product of two functions is not the product of the derivatives of each function. In symbols this can be written 

\begin{center}
$\left (f g\right )^{ \prime } \neq f^{ \prime } g^{ \prime }$
\end{center}\par
\textbf{Theorem} 

If $f$ and $g$ are both differentiable then $\frac{d}{d x} \left [f (x) g (x)\right ] =f (x) \frac{d}{d x} \left [g (x)\right ] +g (x) \frac{d}{d x} \left [f (x)\right ]$ 



The Product rule is often seen in an abbreviated form as
\begin{equation*}\left (u v\right )^{ \prime } =u v^{ \prime } +v u^{ \prime }
\end{equation*}

In words the Product Rule states that \textit{the derivative of the product of two functions is the first function times the derivative of the second function plus the second function times the derivative of the first function.} 



%-----------------------------------------------
\subsection{The Quotient Rule}
Let $u =f (x)$ and $v =g (x)$ be differentiable functions of $x$ then we can show that
\begin{equation*}\frac{d}{d x} \genfrac{[}{]}{}{}{f (x)}{g (x)} =\frac{g (x) \frac{d}{d x} \left [f (x)\right ] -f (x) \frac{d}{d x} \left [g (x)\right ]}{\left [g (x)\right ]^{2}}
\end{equation*}or in abbreviated form
\begin{equation*}\genfrac{(}{)}{}{}{u}{v}^{ \prime } =\frac{v u^{ \prime } -u v^{ \prime }}{v^{2}}
\end{equation*}or in words \textit{the derivative of a quotient is the denominator times the derivative of the numerator minus the numerator times the derivative of the denominator, all divided by the square of the denominator.}


%-----------------------------------------------
\subsection{Chain Rule}
When functions are combined with other functions, they are often called composite functions. These require special treatment when differentiating.

Let $f (x) =x^{2}$ and $g (x) =2 x +1$ then $\left (f \circ g\right )$ This means $f$ `composed of' $g$ is $f \left (g \left (x\right )\right ) =f (2 x +1) =\left (2 x +1\right )^{2}$. 

Also, $g$ `composed of' $f$ would be: $\left (g \circ f\right ) (x) =g \left (x^{2}\right ) =2 \left (x^{2}\right ) +1 =2 x^{2} +1$. 

The differentiation rules we have met so far allow us to differentiate pairs of functions that have been added, subtracted, multiplied or divided. They do not allow us to differentiate an expression that is made from a function that is within another function.

The following are all examples of composite functions. 
\begin{enumerate}
\item We can differentiate $x^{2}$ but we can't use the same procedure to differentiate $\left (1 -x\right )^{2}$. Here we can imagine if  $f (x) =x^{2}$ and $g (x) =1 -x$ then $\left (f \circ g\right ) (x) =f (1 -x) =\left (1 -x\right )^{2}$.

\item We can differentiate $\frac{1}{x^{2}}$ but we can't use the same procedure to differentiate $\frac{1}{x^{2} +1}$. 

\item We can differentiate e$^{x}$ but we can't use the same procedure to differentiate $e^{x^{2}}$. 
\end{enumerate}

A name often used for functions of this type is \emph{function of a function.} 

Once we recognise we are dealing with a composite function we need a procedure to differentiate it. You will find that you are far more likely to be required to differentiate a composite function in a practical situation than a simple one. It can be proved that the derivative of the composite function $f \circ g$ is the product of the derivatives of $f$ and $g$. This important rule is given the name the \emph{Chain Rule}. A substitution method is often used to add clarity to the differentiation process. 

Let
$y =u^{2}$ and let $u =1 -x$. Then $\frac{d y}{d u} =2 u$ and $\frac{d u}{d x} = -1$. Now $\frac{d y}{d x} =\frac{d y}{d u} \cdot \frac{d u}{d x} =2 u \times ( -1) = -2 u = -2 (1 -x) =2 (x -1)$. 

The Leibniz form of the Chain Rule $\displaystyle \frac{d y}{d x} =\frac{d y}{\cancel{d u}} \frac{\cancel{d u}}{d x}$ is what gives the rule its name. Because of the apparent cancelling it is particularly
easy to learn in this form. 

As an aside let us verify the rule for this example. Given
$y =(1 -x)^{2}$. We will expand the right hand side of the equation. It
becomes $y =x^{2} -2 x +1$. So $y^{ \prime } =2 x -2 =2 (x -1)$ as before. 

Using function notation the Chain Rule states: If $f$ and $g$ are both differentiable and $F =f \circ g$ is the\ composite function $F (x) =f (g (x))$, then $F$ is differentiable and $F^{ \prime } =f^{ \prime } (g (x)) g^{ \prime } (x)$. 

\subsection*{A Comment on the Leibniz form of the Chain Rule}
$\frac{d y}{d x} =\frac{d y}{d u} \cdot \frac{d u}{d x}$ gives the impression that the $d u$ could cancel but remember we have not defined $d u$. We have defined $\frac{d y}{d u}$ as the rate of change of $y$ with respect to $u$ and $\frac{d u}{d x}$ as the rate of change of $u$ with respect to $x$. However the apparent cancelling helps us to remember the way the differentials
are arranged. it also helps us to accept the extension of the Chain Rule to cover a function of a function of
a function etc. e.g.
\begin{align*}\text{Let}y &  = f (u)\text{,}u =g (v)\text{and}v =h (x) \\
\text{Then}\frac{d y}{d x} &  = \frac{d y}{d u} \cdot \frac{d u}{d v} \cdot \frac{d v}{d x}\end{align*}

\textbf{Example} 

Find $F^{ \prime } (x)$ when $F (x) =\frac{1}{x^{2} +1}$. 

Using the function notation 

$F (x) =\left (f \circ g\right ) (x) =f (g (x))$ where $f (u) =u^{ -1}$ and $g (x) =x^{2} +1$
\begin{equation*}f^{ \prime } (u) = -u^{ -2}\text{and}g^{ \prime } (x) =2 x
\end{equation*}
and 
\begin{align*}F^{ \prime } (x) &  = f^{ \prime } (g (x)) g^{ \prime } (x) \\
 &  = \frac{ -1}{(x^{2} +1)^{2}} \cdot 2 x \\
 &  = \frac{ -2 x}{\left (x^{2} +1\right )^{2}}\end{align*}
Using the Leibniz notation let $u =x^{2} +1$ and $y =u^{ -1}$ then
\begin{align*}F^{ \prime } (x) &  = \frac{d y}{d u} \frac{d u}{d x} = -u^{ -2} \left (2 x\right ) \\
 &  = \frac{ -1}{\left (x^{2} +1\right )^{2}} \left (2 x\right ) =\frac{ -2 x}{\left (x^{2} +1\right )^{2}}\end{align*}

To use the method we need to bring a new variable into the problem we are trying to solve. It
is recommended that you use the variable $u$ wherever possible so that you follow through using a pattern you are familiar with. 

In summary: if $g$ is differentiable at $x$ and $f$ is differentiable at $g (x)$, then the composite function $F =f \circ g$ defined by $F (x) =f (g (x))$ is differentiable at $x$ and $F^{ \prime }$ is given by the product
\begin{tcolorbox}
	\begin{center}
		$F^{ \prime } (x) =f^{ \prime } (g (x)) \cdot g^{ \prime } (x)$
	\end{center}
\end{tcolorbox}
In Leibniz notation, if $y =f (u)$ and $u =g (x)$ are both differentiable functions, then
\begin{tcolorbox}
	\begin{center}
$\frac{d y}{d x} =\frac{d y}{d u} \cdot \frac{d u}{d x}$
	\end{center}
\end{tcolorbox}

The Chain Rule will be found in many situations where functions are added, subtracted, multiplied or divided. As an example we will focus on combining the Chain Rule with the Product Rule, however any combination of these rules could be found in a problem. 

\textbf{Example} Differentiate $x e^{ -x^{2}}$. 

We can see that there is a product of two functions present in this example, i.e. $f (x) =x$ and $g (x) =e^{ -x^{2}}$. Also $g (x)$ is a composite function. 

We have from the Product Rule
\begin{equation*}\left (f g\right )^{ \prime } =f g^{ \prime } +g f^{ \prime }
\end{equation*}

By inspection we can see that of the four expressions on the right side of this equation $f$, $g$ and $f^{ \prime }$ can be put into the equation immediately and only $g^{ \prime }$ requires some effort to be worked out. 

$g (x)$ is a composite function so $g^{ \prime } (x)$ is computed using the Chain Rule. 

Let $u = -x^{2}$ then $\frac{d u}{d x} = -2 x$. Also $g (u) =e^{u}$ so $\frac{d g}{d u} =e^{u}$.
\begin{align*}\frac{d g}{d x} &  = \frac{d g}{d u} \frac{d u}{d x} \\
 &  = e^{u} \cdot  -2 x \\
 &  = e^{ -x^{2}} \cdot  -2 x \\
 &  =  -2 x\; e^{ -x^{2}} \\
\text{So}g^{ \prime } (x) &  =  -2 x\; e^{ -x^{2}}\end{align*}

Putting this all together
\begin{align*}\left (f g\right )^{ \prime } &  = f g^{ \prime } +g f^{ \prime } \\
 &  = x \cdot  -2 x\; e^{ -x^{2}} +e^{ -x^{2}} \cdot 1 \\
 &  = e^{ -x^{2}} \left [1 -2 x^{2}\right ]\end{align*}




%---------------------------------------------------
% Parametric Differentiation
%---------------------------------------------------
\section{Parametric Differentiation}
Parametric Curves $x$ and $y$ are both given as functions of a third variable $t$ (called the \emph{parameter}). Let the equations be
\begin{equation*}x =f (t)\text{ and }y =g (t)
\end{equation*}

Each value of $t$ gives a point $(x ,y)$. As $t$ varies the point $(x ,y) =(f (t) ,g (t))$ traces out a curve in the coordinate plane called a \emph{parametric curve}. 


\subsection*{Using the Chain Rule to find the Derivative $\frac{d y}{d x}$}
Given\ the parametric equations $x =f (t)$ and $y =g (t)$ define a parametric curve. If $f$ and $g$ are both differentiable the Chain Rule gives
\begin{equation*}\frac{d y}{d t} =\frac{d y}{\cancel{d x}} \cdot \frac{\cancel{d x}}{d t}
\end{equation*}

provided $y$ is also a differentiable function of $x$. So provided $\frac{d x}{d t} \neq 0$
\begin{align*}\frac{d y}{d x} &  = \frac{d y}{d t} \div \frac{d x}{d t} \\
 &  = \frac{d y}{d t} \times \frac{d t}{d x}\end{align*}



\section{Related Rates}
The concept of related rates is best understood by exploring some examples. 

\subsubsection{Example 1}
Air is being pumped into a spherical balloon so that its volume is increasing at a rate of 100 $cm^{3}$/$\mbox{s}$. How fast is the radius of
the balloon increasing when the diameter is 50 $\mbox{cm}$? 

You have to find the related rates in the question. Let
$V$ be the volume and $r$ be the radius at time $t$. The volume is increasing at the rate of 100 $cm^{3}$/$\mbox{s}$ so $\frac{d V}{d t} =100$. The question asks how fast is the radius increasing. In
other words what is $\frac{d r}{d t}$? 

\textbf{Solution:} 

We need a formula that connects $V$ and $r$ if we are to find the relationship between $\frac{d V}{d t}$ and $\frac{d r}{d t}$. 

The formula
\begin{equation*}V =\frac{4}{3} \pi  r^{3}
\end{equation*}

is of the form $V =f (r)$ so we can find $\frac{d V}{d r}$
\begin{equation}\frac{d V}{d r} =4 \pi  r^{2}\tag{1}
\end{equation}

From the Chain Rule we can write
\begin{equation}\frac{d V}{d t} =\frac{d V}{d r} \cdot \frac{d r}{d t}\tag{2}
\end{equation}

Substituting $\frac{d V}{d t} =100$ and $\frac{d V}{d r} =4 \pi  r^{2}$ in equation (2) we get
\begin{align*}100 &  = 4 \pi  r^{2} \cdot \frac{d r}{d t} \\
\frac{d r}{d t} &  = \frac{100}{4 \pi  r^{2}} \\
 &  = \frac{25}{\pi  r^{2}}\end{align*}

Now we substitute $r =25$. (Diameter $ =50$ so radius $ =25$)
\begin{align*}\frac{d r}{d t}_{r =25} &  = \frac{25}{\pi  \times 25^{2}} \\
 &  = \frac{1}{25 \pi }\end{align*}

So the radius is increasing at the rate of $\frac{1}{25 \pi } \mbox{cm}$/$\mbox{s}$ 

\subsubsection{Example 2}
A ladder $5$ $\mbox{m}$ long rests against a vertical wall. If
the bottom of the ladder slides away from the wall at the rate of $0.5$ $\mbox{m}$/$\mbox{s}$ how fast is the top of the ladder sliding down the wall when the bottom
of the ladder is $3$ $\mbox{m}$ from the wall? 

\textbf{Solution:} 

Let the origin be placed at the corner where the wall meets the floor, let $x$ be the distance of the foot of the ladder from the wall and let $y$ be the distance of the top of the ladder from the corner. The ladder forms a right
angled triangle whose sides are $x$, $y$ and with hypotenuse $5$. We are given that $\frac{d x}{d t} =0.5$ and are asked to find $\frac{d y}{d t}$ when $x =3$. 

Pythagoras theorem gives
\begin{equation}x^{2} +y^{2} =5^{2}\tag{1}
\end{equation}

Differentiate equation (1) with respect to $t$
\begin{equation*}2 x \frac{d x}{d t} +2 y \frac{d y}{d t} =0
\end{equation*}

Solve for $\frac{d y}{d t}$
\begin{equation*}\frac{d y}{d t} = -\frac{x}{y} \cdot \frac{d x}{d t}
\end{equation*}

Using Pythagoras theorem when $x =3$ and the hypotenuse $ =5$, $y =4$ 

Substitute $\frac{d x}{d t} =0.5$, $x =3$ and $y =4$
\begin{align*}\frac{d y}{d t} &  =  -\frac{3}{4} \cdot 0.5 \\
 &  =  -0.375\end{align*}

The top of the ladder is moving vertically downwards at the rate of 0.375 $\mbox{m}$/$\mbox{s}$ 

\subsubsection{Example 3}
A water tank has the shape of an inverted circular cone with a base radius of $2$ $\mbox{m}$ and height of $4$ $\mbox{m}$. If water is being pumped into
the tank at a rate of $2$ $\mathrm{m}^{3}$/$\mbox{min}$ find the rate at which the water level is rising when the water is $3$ $\mbox{m}$ deep. 

\textbf{Solution:} 

Let
$V$, $r$ and $h$ be the volume of water the radius of the surface and the height at time $t$. We are given
\begin{equation*}\frac{d V}{d t} =2\text{}\mathrm{m}^{3}/\mbox{min}
\end{equation*}

We are asked to find $\frac{d h}{d t}$ when $h =3$. 

Draw a diagram to show that the relationship between $r$ and $h$ can be found by similar triangles.
\begin{align}\frac{r}{h} &  = \frac{2}{4} \nonumber  \\
r &  = \frac{h}{2} \tag{1}\end{align}

The formula for the volume is
\begin{equation}V =\frac{1}{3} \pi  r^{2} h\tag{2}
\end{equation}

Substituting equation (1) in equation (2)
\begin{align*}V &  = \frac{1}{3} \pi  \genfrac{(}{)}{}{}{h}{2}^{2} h \\
 &  = \frac{\pi }{12} h^{3}\end{align*}

Differentiate with respect to $t$
\begin{align*}\frac{d V}{d t} &  = \frac{\pi }{12} \cdot 3 h^{2} \cdot \frac{d h}{d t} \\
 &  = \frac{\pi }{4} h^{2} \cdot \frac{d h}{d t}\end{align*}

So
\begin{equation*}\frac{d h}{d t} =\frac{4}{\pi  h^{2}} \cdot \frac{d V}{d t}
\end{equation*}

Substitute $h =3$ and $\frac{d V}{d t} =2$
\begin{align*}\frac{d h}{d t} &  = \frac{4}{\pi  \left (3\right )^{3}} \cdot 2 \\
 &  = \frac{8}{9 \pi }\end{align*}

The water level is rising at the rate of $\frac{8}{9 \pi }$ $\mbox{m}$/$\mbox{min}$. 

%---------------------------------------------------
% Optimisation
%---------------------------------------------------
\section{Optimisation}
This group of problems is often called \emph{optimisation} problems. We are finding
the optimum (best) way to do something. We could be finding for example the maximum size or maximum velocity
or we could be finding the minimum cost etc. 

\subsection*{Method 1}
Find $f^{ \prime } (x)$. Then find values of $x$ that are solutions of the equation $f^{ \prime } (x) =0$. Test values near each value of $x$ to find whether the slope of the tangent goes from positive through zero to negative (local maximum) or negative through zero
to positive (local minimum). 

\subsection*{Method 2}
Find $f^{ \prime } (x)$ and $f^{ \prime  \prime } (x)$. Then find values of $x$ that are solutions of the equation $f^{ \prime } (x) =0$. Substitute each of these values of $x$ in $f^{ \prime  \prime } (x)$. If $f^{ \prime  \prime } (x) >0$ the point is a local minimum. If $f^{ \prime  \prime } (x) <0$ the point is a local maximum. Also if $f^{ \prime  \prime } (x) =0$ the point is a point of inflection.\vspace{1cm} 

\subsection*{Steps to Solve an Optimisation Problem}
\begin{enumerate}
\item Understand the problem. What is the unknown? What
are the given quantities? What are the given conditions? 

\item Draw
a diagram. This can be a very useful step. 

\item Assign
symbols to the quantities. (Both the known and unknown quantities.) 

\item Express the symbol for the
unknown quantity in terms of the symbols for the known quantities. 

\item Use the facts of the problem
to reduce the expression in step 4 until is becomes a relationship between the unknown quantity and \emph{one} of the known quantities.


\item Use one of the methods given above to find the maximum or minimum value.\vspace{1cm}
\end{enumerate}



%---------------------------------------------------
% Chapter Exercises in a separate file
%---------------------------------------------------
% \section{Chapter Exercises}
% \subimport{}{differentiationExercises}
