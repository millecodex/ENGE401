\chapter{Trigonometry of Angles}

In chapter 2 the three trigonometric functions (sine, cosine and tangent) were viewed as functions of real numbers. \ In
this chapter they will be viewed as functions of angles. \ The traditional approach will usually firstly view
sine, cosine and tangent as functions of angles. \ In fact many users of trigonometry will have only met the trigonometry
of angles. 

\section{3.1 The Measurement of Angles}
Readings pp 473-480 

It should come as no surprise that the convention for measuring angles should be the same as for measuring
the distance around the perimeter of the unit circle.. \ An angle is measured in degrees and is the amount of
rotation between two rays about a vertex. \ If the vertex is placed at the point $\left (0 ,0\right )$ and one ray is placed along the positive $x$-axis then we let the second ray go through the point $P (x ,y)$ on the unit circle so that the relationship between the angle and $t$ (see chapter $2$) can be established. \ The convention is that the positive direction for measuring
angles is anticlockwise and the negative direction for measuring angles is clockwise. \ You will be aware that
one complete cycle measures 360$\mbox{{\ensuremath{{}^\circ}}}$. \ One degree therefore is $\frac{1}{360}$\ of one complete cycle. \ In this subject the
word revolution is often used instead of cycle. \ Other terms with which you might be familiar are "a quarter
turn" for $90 \mbox{{\ensuremath{{}^\circ}}}$ and "a half turn" for $180 \mbox{{\ensuremath{{}^\circ}}}$. 

If you draw a unit circle (centre $\left (0 ,0\right )$ radius $1$) then you can draw the ray through any terminal point and show the relationship between $t$ and the angle at the vertex $\left (0 ,0\right )$. \ The diagrams on p
474 of the textbook show this relationship. 

Definition: If the unit circle is drawn and the distance $t$ is measured around the perimeter from $\left (1 ,0\right )$ to the point $P (x ,y)$ then we say the angle is measured as $t$ \emph{radians}. 

The abbreviation for radians is $\mbox{rad}$. \ This abbreviation will be used in the examples.
\ The textbook uses the term "angles in standard position" to describe this situation. \ We
will not define this term unless it is unavoidable. 

Using the language of geometry. \ The
unit circle is cut by two rays one through the points $\left (0 ,0\right )$ and $\left (1 ,0\right )$ and the other through the points $\left (0 ,0\right )$ and $P \left (x ,y\right )$.  
%TCIMACRO{\TeXButton{Start Two Columns}{\columnsep =30pt
% \begin {multicols}{2}}}%
%BeginExpansion
\columnsep =30pt
\begin {multicols}{2}
%EndExpansion
 

   
\setlength\fboxrule{0in}\setlength\fboxsep{0.2in}\fcolorbox[HTML]{000000}{FFFFFF}{\includegraphics[ width=2.469in, height=2.4275in,]{L4SZ281A}
}


$t$ is referred to as "the length of the arc".\vspace{1cm} 

\ The
angle $\theta $ between the two rays is referred to as "the angle \emph{subtended} at the point $\left (0 ,0\right )$". 

We will often leave out the word "subtended"
however this is because it is implied and we really should have used it. 


%TCIMACRO{\TeXButton{End Two Columns}{\end {multicols}}}%
%BeginExpansion
\end {multicols}
%EndExpansion


We can say therefore that the angle measured in radians is related to the same angle measured in degrees. \ the
relationship between these two measures must be understood and must be able to be derived. 

\subsection{Relationship between Degrees and Radians}
One complete revolution is 360$\mbox{{\ensuremath{{}^\circ}}}$ if the angle is measured in degrees and $2 \pi $ if the angle is measured in radians. \ So
\begin{align*}2 \pi \text{}\mbox{rad} &  = & 360 \mbox{{\ensuremath{{}^\circ}}} \\
\text{or\ \ \ \ \ }\pi \text{}\mbox{rad} &  = & 180 \mbox{{\ensuremath{{}^\circ}}}\end{align*}

You should derive this formula whenever you are asked to convert degrees to radians or radians
to degrees 

\subsubsection{Example 1}
(a) Convert 36$\mbox{{\ensuremath{{}^\circ}}}$ to radians 

(b) Convert $\frac{\pi }{3}$ $\mbox{rad}$ to degrees 

(c) Convert $1$ $\mbox{rad}$ to degrees 

(a)
\begin{align*}180 \mbox{{\ensuremath{{}^\circ}}} &  = & \pi \text{}\mbox{rad} \\
1 \mbox{{\ensuremath{{}^\circ}}} &  = & \frac{\pi }{180}\text{}\mbox{rad} \\
36 \mbox{{\ensuremath{{}^\circ}}} &  = & \frac{\pi }{180} \times 36 =\frac{\pi }{5}\text{}\mbox{rad}\end{align*}

(b)
\begin{align*}\pi  \mbox{rad} &  = & 180 \mbox{{\ensuremath{{}^\circ}}} \\
\frac{\pi }{3} \mbox{rad} &  = & \frac{180 \mbox{{\ensuremath{{}^\circ}}}}{3} =60 \mbox{{\ensuremath{{}^\circ}}}\end{align*}

(c)
\begin{align*}\pi  \mbox{rad} &  = & 180 \mbox{{\ensuremath{{}^\circ}}} \\
1 \mbox{rad} &  = & \frac{180 \mbox{{\ensuremath{{}^\circ}}}}{\pi } \\
 &  \approx  & 57.29577951 \mbox{{\ensuremath{{}^\circ}}} \\
 &  \approx  & 57.3 \mbox{{\ensuremath{{}^\circ}}}\end{align*}

Note the similarity between the answers to (b) and (c). \ This
is because $\frac{\pi }{3} =1.047197551$ so you would expect the values in degrees to be similar. 

Example 1 pp 474-475 provides an additional
example. 

With this terminology we leave out the word measured when we talk about measuring angles. \ We
say "the angle is $60 \mbox{{\ensuremath{{}^\circ}}}$" when we mean it has been measured as $60 \mbox{{\ensuremath{{}^\circ}}}$ or "the angle is $\frac{\pi }{3}$" to mean it has been measured as $\frac{\pi }{3}$ $\mbox{rad}$. \ Notice we always put in the degree symbol and
often omit the units when the angle is measured in radians. \ Should units be omitted assume the angle is measured
in radians. \ We often use the Greek symbol $\theta $\ for the angle subtended at the centre of the unit circle so $\theta  =60 \mbox{{\ensuremath{{}^\circ}}}$ or $\theta  =\frac{\pi }{3}$ are further examples of terminology that is commonly used.

\subsection{Length of a Circular Arc}
Let $\theta $ be the angle subtended at the centre for the ends of an arc of any circle then the fraction of the circumference of the
circle is $\frac{\theta }{2 \pi }$ if $\theta $ is measured in radians and $\frac{\theta }{360 \mbox{{\ensuremath{{}^\circ}}}}$ if $\theta $ is measured in degrees. 

The length of the circumference of any circle whose radius is $r$ is $2 \pi  r$. 

If $\theta $ is measured in radians
\begin{equation*}\text{Length of arc} =\frac{\theta }{2 \pi } \times 2 \pi  r =\theta  r\text{or}r \theta 
\end{equation*}

If $\theta $ is measured in degrees
\begin{equation*}\text{Length of arc} =\frac{\theta }{360 \mbox{{\ensuremath{{}^\circ}}}} \times 2 \pi  r
\end{equation*}

The simplicity of the first formula shows why working with radians is preferred. 

\subsubsection{Example 3}
Find the length of an arc that subtends an angle of 45$\mbox{{\ensuremath{{}^\circ}}}$ at the centre of a circle whose radius is 9 $\mbox{cm}\text{.}$ 

Method 1:
\begin{align*}\text{Length of arc} &  = & \frac{\theta }{360 \mbox{{\ensuremath{{}^\circ}}}} \times 2 \pi  r \\
 &  = & \frac{45}{360} \times 2 \pi  \times 9 \\
 &  = & 2.25 \pi \text{}\mbox{cm} \\
 &  \approx  & 7.07\text{}\mbox{cm}\end{align*}

If an exact answer is required you should leave the answer as $2.25 \pi $ $\mbox{cm}$. \ (Or $\frac{9 \pi }{4}$ $\mbox{cm}\text{.}$) 

Method 2: Change degrees to radians first
\begin{align*}180 \mbox{{\ensuremath{{}^\circ}}} &  = & \pi \text{}\mbox{rad} \\
1 \mbox{{\ensuremath{{}^\circ}}} &  = & \frac{\pi }{180} \\
45 \mbox{{\ensuremath{{}^\circ}}} &  = & \frac{\pi }{180} \times 45 \\
 &  = & \frac{\pi }{4}\end{align*}


\begin{align*}\text{Length of arc} &  = & r \theta  \\
 &  = & 9 \times \frac{\pi }{4} =\frac{9 \pi }{4}\text{}\mbox{cm}\end{align*}

Example 4 on p 478 provides an additional example. 

\subsection{Area of Circular Sector}
It is assumed you know how to describe and visualise a sector of a circle and that you know that the area of a circle whose radius is $r$ is $\pi  r^{2}$. \ Continuing the logic above 

If $\theta $ is measured in radians
\begin{equation*}\text{Area of sector} =\frac{\theta }{2 \pi } \times \pi  r^{2} =\frac{1}{2} r^{2} \theta 
\end{equation*}

If $\theta $ is measured in degrees
\begin{equation*}\text{Area of sector} =\frac{\theta }{360 \mbox{{\ensuremath{{}^\circ}}}} \times \pi  r^{2}
\end{equation*}

\subsubsection{Example 4}
Find the area of a sector with a central angle of 45$\mbox{{\ensuremath{{}^\circ}}}$ for a circle whose radius is 4 $\mbox{cm}$. 

Method 1
\begin{align*}\text{Area of sector} &  = & \frac{45 \mbox{{\ensuremath{{}^\circ}}}}{360 \mbox{{\ensuremath{{}^\circ}}}} \times \pi  \times 4^{2} \\
 &  = & 2 \pi \text{}cm^{2} \\
 &  \approx  & 6.28\text{}cm^{2}\end{align*}

Method 2 \ As above
\begin{align*}180 \mbox{{\ensuremath{{}^\circ}}} &  = & \pi \text{}\mbox{rad} \\
1 \mbox{{\ensuremath{{}^\circ}}} &  = & \frac{\pi }{180} \\
45 \mbox{{\ensuremath{{}^\circ}}} &  = & \frac{\pi }{180} \times 45 \\
 &  = & \frac{\pi }{4}\end{align*}
\begin{align*}\text{Area of sector} &  = & \frac{1}{2} r^{2} \theta  \\
 &  = & \frac{1}{2} \times 4^{2} \times \frac{\pi }{4} \\
 &  = & 2 \pi \text{}cm^{2}\end{align*}

Example 5 on p 478 provides another example. 

\subsection{Circular Motion}
We use the term circular motion to describe a particle moving in a circle around a vertex. \ We can
either describe the speed of the particle as angular speed or linear speed. \ Let $\theta $ be measured in radians and time $t$ be measured in seconds, then the angular speed ($\omega $) (measured in $\mbox{rad}$/$\mbox{s}$) is
\begin{equation}\omega  =\frac{\theta }{t}\tag{1}
\end{equation}

If $r$ ($\mbox{cm}$) is the radius of the circle so that the length of the arc, $s =r \theta $ then the linear speed ($v$) (measured in $\mbox{cm}$/$\mbox{s}$) is
\begin{equation}v =\frac{s}{t}\tag{2}
\end{equation}

Because this is the motion of the same particle $v$ and $\omega $ are connected.
\begin{align}\text{From (2)\ \ \ \ }v &  = & \frac{s}{t} =\frac{r \theta }{t} \nonumber  \\
\text{But from (2)\ \ \ }\frac{\theta }{t} &  = & \omega \text{\ \ so} \nonumber  \\
v &  = & r \times \frac{\theta }{t} =r \omega  \tag{3}\end{align}

So this shows how to obtain $v$ from $\omega $ or $\omega $ from $v$. 

\subsubsection{Example 5}
Find the angular and linear speed for a tip on the end of a fan blade whose radius is $600$ $\mbox{mm}$ that spins at $500$ rpm.
\begin{equation*}\text{Angular speed}\omega  =\frac{\theta }{t}
\end{equation*}

How do we obtain $\theta $ and $t$ from $500$ rpm?
\begin{align*}500\text{rpm} &  = & \frac{500}{60}\text{rev/}\mbox{s} \\
 &  = & \frac{25}{3}\text{rev/}\mbox{s}\end{align*}


\begin{align*}1\text{rev} &  = & 2 \pi \text{rad} \\
\frac{25}{3}\text{rev} &  = & \frac{25}{3} \times 2 \pi  =\frac{50 \pi }{3}\end{align*}

So if $t =1$ $\theta  =\frac{50 \pi }{3}$ 


\begin{align*}\omega  &  = & \frac{50 \pi }{3} \div 1 \\
 &  = & \frac{50 \pi }{3}\text{rad/}\mbox{s}\end{align*}


\begin{align*}\text{Linear speed}v &  = & \frac{s}{t} \\
\text{But\ \ \ }s &  = & r \theta  =600 \times \frac{50 \pi }{3} \\
\text{So\ \ \ }v &  = & 10000 \pi \text{}\mbox{mm}\text{/}\mbox{s} \\
 &  = & 1000 \pi \text{}\mbox{cm}\text{/}\mbox{s} \\
 &  = & 10 \pi \text{}\mbox{m}\text{/}\mbox{s}\end{align*}

Examples 6 and 7 pp 479-480 provide additional examples. 

\subsection{Exercises}
The following exercises from pp 480-483 have been covered in this section: 

Find the radian measure of the angle with the given degree measurements. 


\begin{description}
\item [1.]   
%TCIMACRO{\TeXButton{Start Two Columns}{\columnsep =30pt
% \begin {multicols}{2}}}%
%BeginExpansion
\columnsep =30pt
\begin {multicols}{2}
%EndExpansion
 $36 \mbox{{\ensuremath{{}^\circ}}}$ 

\item [3.]
$ -480 \mbox{{\ensuremath{{}^\circ}}}$ 
%TCIMACRO{\TeXButton{End Two Columns}{\end {multicols}}}%
%BeginExpansion
\end {multicols}
%EndExpansion
 

\item [5.]
%TCIMACRO{\TeXButton{Start Two Columns}{\columnsep =30pt
% \begin {multicols}{2}}}%
%BeginExpansion
\columnsep =30pt
\begin {multicols}{2}
%EndExpansion
 $60 \mbox{{\ensuremath{{}^\circ}}}$ 

\item [7.]
$ -135 \mbox{{\ensuremath{{}^\circ}}}$ 
%TCIMACRO{\TeXButton{End Two Columns}{\end {multicols}}}%
%BeginExpansion
\end {multicols}
%EndExpansion
 \end{description}

Find the degree measure of the angle with the given radian measure. 


\begin{description}
\item [9.]   
%TCIMACRO{\TeXButton{Start Two Columns}{\columnsep =30pt
% \begin {multicols}{2}}}%
%BeginExpansion
\columnsep =30pt
\begin {multicols}{2}
%EndExpansion
 $\frac{3 \pi }{4}$ 

\item [11.] $\frac{5 \pi }{6}$ 
%TCIMACRO{\TeXButton{End Two Columns}{\end {multicols}}}%
%BeginExpansion
\end {multicols}
%EndExpansion
 

\item [13.]
%TCIMACRO{\TeXButton{Start Two Columns}{\columnsep =30pt
% \begin {multicols}{2}}}%
%BeginExpansion
\columnsep =30pt
\begin {multicols}{2}
%EndExpansion
 $ -1.5$ 

\item [15.] $ -\frac{\pi }{12}$ 
%TCIMACRO{\TeXButton{End Two Columns}{\end {multicols}}}%
%BeginExpansion
\end {multicols}
%EndExpansion
 

\item [41.]
%TCIMACRO{\TeXButton{Start Two Columns}{\columnsep =30pt
% \begin {multicols}{2}}}%
%BeginExpansion
\columnsep =30pt
\begin {multicols}{2}
%EndExpansion
 Find the length of the arc $s$ in the figure. \\\relax The radius is $5$. 

\item    
\setlength\fboxrule{0in}\setlength\fboxsep{0.2in}\fcolorbox[HTML]{000000}{FFFFFF}{\includegraphics[ width=1.8057in, height=1.8507in,]{L4SZ281B}
}


\item [43.] Find the radius $r$ of the circle in the figure. 

\item    
\setlength\fboxrule{0in}\setlength\fboxsep{0.2in}\fcolorbox[HTML]{000000}{FFFFFF}{\includegraphics[ width=1.8196in, height=1.8438in,]{L4SZ281C}
}
%TCIMACRO{\TeXButton{End Two Columns}{\end {multicols}}}%
%BeginExpansion
\end {multicols}
%EndExpansion
 

\item [45.]
Find the length of an arc that subtends a central angle of $2 \mbox{rad}$ in a circle of radius 2mi. 

\item [47.]
An arc of length $100 \mbox{m}$ subtends a central angle \ in
a circle of radius $50 \mbox{m}$. \ Find the measure of $\theta $\ in degrees and in radians. 

\item [49.]
Find the radius of the circle if an arc of length $6 \mbox{m}$ on the circle subtends a central angle of $\pi /6$ rad. 

\item [51] Pittsburgh, Pennsylvania
and Miami, Florida lie approximately on the same meridian. \ Pittsburgh has a latitude of $40.5 \mbox{{\ensuremath{{}^\circ}}}$ N and Miami is $25.5 \mbox{{\ensuremath{{}^\circ}}}$ N. Find the distance between these two cities. \ (The
radius of the earth is $3960 \mbox{mi}\text{.}$) 

\item [53.] Find the distance
the earth travels in one day in its path around the sun. \ Assume the year has $365$ days and that the path of the earth around the sun is a circle of radius $93$ million miles. 

\item [55.] Find
the distance along an arc on the surface of the earth that subtends an angle of 1 minute. \ ($1$ minute = $\frac{1}{60}$ degree). \ This distance is called a nautical mile. \ The
radius of the earth is $3960 \mbox{mi}\text{.}$ 

\item [57.] Find the area
of a sector with a central angle $1 \mbox{rad}$ in a circle of radius $10 \mbox{m}$. 

\item [59.]
The area of a sector of a circle with a central angle of $2 \mbox{rad}$ is $16 \mathrm{m}^{2}$. \ Find the radius of the circle. 

\item [61.]
The area of the circle is $72 cm^{2}$. \ Find the area of a sector of the circle that subtends an angle of $\pi /6 \mbox{rad}\text{.}$ 

\item [63.]   
%TCIMACRO{\TeXButton{Start Two Columns}{\columnsep =30pt
% \begin {multicols}{2}}}%
%BeginExpansion
\columnsep =30pt
\begin {multicols}{2}
%EndExpansion
 A winch of radius $2 \mbox{ft}$ is used to lift heavy loads. \ If the winch makes
$8$ revolutions in $15 \mbox{s}$, find the speed at which the load is rising. 

\item
\setlength\fboxrule{0in}\setlength\fboxsep{0.2in}\fcolorbox[HTML]{000000}{FFFFFF}{\includegraphics[ width=1.0646in, height=2.5849in,]{L4SZ281D}
}
%TCIMACRO{\TeXButton{End Two Columns}{\end {multicols}}}%
%BeginExpansion
\end {multicols}
%EndExpansion
  

\item [65.] A radial saw has a blade with a $6 \mbox{in}$ radius. \ Suppose that the blade spins at a speed
of $1000$ rpm. 

\item [(a)] Find the angular
speed of the blade in $rad/\mbox{min}\text{.}$ 

\item [(b)] Find the linear
speed of the sawteeth in $ft/\mbox{s}$. 

\item [67.] The wheels of a car have
a radius $11 \mbox{in}$ and are rotating at $600$ rpm. \ Find the speed of the car in $mi/\mbox{h}$. 

\item [69.] To measure the speed of
a current scientists place a paddle wheel in the stream and observe the rate at which it rotates. \ If the paddle
wheel has a radius $0.20 \mbox{m}$ and rotates at $100$ rpm, find the speed of the current in $\mathrm{m}/\mbox{s}$. \end{description}


%TCIMACRO{\TeXButton{Start Two Columns}{\columnsep =30pt
% \begin {multicols}{2}}}%
%BeginExpansion
\columnsep =30pt
\begin {multicols}{2}
%EndExpansion
 


%TCIMACRO{\TeXButton{End Two Columns}{\end {multicols}}}%
%BeginExpansion
\end {multicols}
%EndExpansion
 

\section{3.2 The Trigonometry of Right Triangles}
Readings pp 483-489 

In this section we will use sine, cosine and tangent. \ You
should have at your fingertips the definitions of sine, cosine and tangent for a right angled triangle:
\begin{equation*}\sin  \theta  =\frac{\text{opposite}}{\text{hypotenuse}}\text{,}\cos  \theta  =\frac{\text{adjacent}}{\text{hypotenuse}}\text{and}\tan  \theta  =\frac{\text{opposite}}{\text{adjacent}}
\end{equation*}

You should be able to solve problems involving right angled triangles. \ A
right angled triangle is uniquely defined, if as well as the right angle, you are given two other facts about the triangle. 


\begin{enumerate}
\item Given one side and one angle 

\item Given two sides \end{enumerate}


To be given two angles does not constitute two other facts as the two angles are complementary so that given one angle the other one is known.



\begin{enumerate}
\item To solve a right triangle given one side and one angle 


\begin{description}
\item [(a)] Sketch. 

\item [(b)]
Fill in the given information and note which sides are "opp", "adj" and "hyp". 

\item [(c)]
Should the other angle be required it is found by subtraction. 

\item [(d)]
Mark the required side, pick the appropriate trigonometric ratio and solve the equation. \end{description}

\item To solve a right triangle given two sides 


\begin{description}
\item [(a)] Sketch. 

\item [(b)]
Fill in the given information. 

\item [(c)] If the third side
is required use Pythagoras theorem. 

\item [(d)] Mark the required
angle, pick the appropriate trigonometric ratio and solve the equation. \ (Use $\sin ^{ -1}$, $\cos ^{ -1}$ or $\tan ^{ -1}\text{.}$) \end{description}\end{enumerate}


\subsection{The Trigonometric Ratios of the Special Angles}
%TCIMACRO{\TeXButton{Start Two Columns}{\columnsep =30pt
% \begin {multicols}{2}}}%
%BeginExpansion
\columnsep =30pt
\begin {multicols}{2}
%EndExpansion
 

   
\setlength\fboxrule{0in}\setlength\fboxsep{0.2in}\fcolorbox[HTML]{000000}{FFFFFF}{\includegraphics[ width=2.3808in, height=2.4284in,]{L4SZ281E}
}


You should be familiar with the 45, 45, 90 triangle and the ratio of the sides. \ Let
the two shorter sides have a length of $1$ then the hypotenuse has a length of $\sqrt{2}$. \\\relax
\begin{align*}1^{2} +1^{2} &  = & 1 +1 =2 \\
\left (\sqrt{2}\right )^{2} &  = & 2 \\
\text{so}1^{2} +1^{2} &  = & \left (\sqrt{2}\right )^{2}\end{align*}


%TCIMACRO{\TeXButton{End Two Columns}{\end {multicols}}}%
%BeginExpansion
\end {multicols}
%EndExpansion



%TCIMACRO{\TeXButton{Start Two Columns}{\columnsep =30pt
% \begin {multicols}{2}}}%
%BeginExpansion
\columnsep =30pt
\begin {multicols}{2}
%EndExpansion
 

   
\setlength\fboxrule{0in}\setlength\fboxsep{0.2in}\fcolorbox[HTML]{000000}{FFFFFF}{\includegraphics[ width=1.8127in, height=2.4284in,]{L4SZ281F}
}


Similarly the equilateral triangle whose sides are 2 units gives rise to the 30, 60, 90 triangle whose sides are 1, $\sqrt{3}$ and 2. \\\relax
\begin{align*}\left (\sqrt{3}\right )^{2} +1^{2} &  = & 3 +1 =4 \\
2^{2} &  = & 4 \\
\text{so}\left (\sqrt{3}\right )^{2} +1^{2} &  = & 2^{2}\end{align*}


%TCIMACRO{\TeXButton{End Two Columns}{\end {multicols}}}%
%BeginExpansion
\end {multicols}
%EndExpansion


The two special right triangles are the "45, 45, 90 triangle" and the "30, 60, 90 triangle". \ Pythagoras
theorem gives the ratio of the sides as 1, 1, $\sqrt{2}$ for the 45$\mbox{{\ensuremath{{}^\circ}}}$ 45$\mbox{{\ensuremath{{}^\circ}}}$ 90$\mbox{{\ensuremath{{}^\circ}}}$ triangle and 1, $\sqrt{3}\text{,}$ 2 for the 30$\mbox{{\ensuremath{{}^\circ}}}$ 60$\mbox{{\ensuremath{{}^\circ}}}$ 90$\mbox{{\ensuremath{{}^\circ}}}$ triangle, which is what we usually remember. 

\subsubsection{Example}
$\sin  45 \mbox{{\ensuremath{{}^\circ}}} =\frac{\text{opp}}{\text{hyp}} =\frac{1}{\sqrt{2}}$ 

We usually rationalise the denominator in expressions like $\frac{1}{\sqrt{2}}$. \ In other words we apply a transformation to remove the $\sqrt{2}$ from the denominator and replace it with a rational number. \ In this case if you multiply
the $\frac{1}{\sqrt{2}}$ by $\frac{\sqrt{2}}{\sqrt{2}}$ you get $\frac{1}{\sqrt{2}} \times \frac{\sqrt{2}}{\sqrt{2}} =\frac{\sqrt{2}}{2}$. 

Another way to achieve this is to start with the triangle 


%TCIMACRO{\TeXButton{Start Two Columns}{\columnsep =30pt
% \begin {multicols}{2}}}%
%BeginExpansion
\columnsep =30pt
\begin {multicols}{2}
%EndExpansion
 

   
\setlength\fboxrule{0in}\setlength\fboxsep{0.2in}\fcolorbox[HTML]{000000}{FFFFFF}{\includegraphics[ width=2.4639in, height=2.4284in,]{L4SZ281G}
}


By Pythagoras theorem
\begin{equation*}\left (\sqrt{2}\right )^{2} +\left (\sqrt{2}\right )^{2} =2 +2 =4 =2^{2}
\end{equation*}


%TCIMACRO{\TeXButton{End Two Columns}{\end {multicols}}}%
%BeginExpansion
\end {multicols}
%EndExpansion
\begin{equation*}\text{So}\sin  45 \mbox{{\ensuremath{{}^\circ}}} =\frac{\sqrt{2}}{2}\text{,}\cos  45 \mbox{{\ensuremath{{}^\circ}}} =\frac{\sqrt{2}}{2}\text{and}\tan  45 \mbox{{\ensuremath{{}^\circ}}} =\frac{\sqrt{2}}{\sqrt{2}} =1
\end{equation*}

If you are asked for the exact values of $\sin  45 \mbox{{\ensuremath{{}^\circ}}}$, $\cos  45 \mbox{{\ensuremath{{}^\circ}}}$ or $\tan  45 \mbox{{\ensuremath{{}^\circ}}}$ you should give these results. 

From the 30, 60, 90 triangle
\begin{align*}\sin  30 \mbox{{\ensuremath{{}^\circ}}} &  = & \frac{1}{2}\text{,}\cos  30 \mbox{{\ensuremath{{}^\circ}}} =\frac{\sqrt{3}}{2}\text{and}\tan  30 \mbox{{\ensuremath{{}^\circ}}} =\frac{1}{\sqrt{3}} =\frac{\sqrt{3}}{3} \\
\sin  60 \mbox{{\ensuremath{{}^\circ}}} &  = & \frac{\sqrt{3}}{2}\text{,}\cos  60 \mbox{{\ensuremath{{}^\circ}}} =\frac{1}{2}\text{and}\tan  60 \mbox{{\ensuremath{{}^\circ}}} =\frac{\sqrt{3}}{1} =\sqrt{3}\end{align*}

\subsection{The Relationship between Degrees and Radians for the Special Right Triangles}
From chapter 7 we showed that
\begin{align*}\sin  \frac{\pi }{4} &  = & \frac{\sqrt{2}}{2}\text{,}\cos  \frac{\pi }{4} =\frac{\sqrt{2}}{2}\text{and}\tan  \frac{\pi }{4} =\frac{\sqrt{2}}{\sqrt{2}} =1 \\
\text{So}\frac{\pi }{4} &  = & 45 \mbox{{\ensuremath{{}^\circ}}}\end{align*}


\begin{align*}\sin  \frac{\pi }{6} &  = & \frac{1}{2}\text{,}\cos  \frac{\pi }{6} =\frac{\sqrt{3}}{2}\text{and}\tan  \frac{\pi }{6} =\frac{1}{\sqrt{3}} =\frac{\sqrt{3}}{3} \\
\text{So}\frac{\pi }{6} &  = & 30 \mbox{{\ensuremath{{}^\circ}}}\end{align*}
\begin{align*}\sin  \frac{\pi }{3} &  = & \frac{\sqrt{3}}{2}\text{,}\cos  \frac{\pi }{3} =\frac{1}{2}\text{and}\tan  \frac{\pi }{3} =\frac{\sqrt{3}}{1} =\sqrt{3} \\
\text{So}\frac{\pi }{3} &  = & 60 \mbox{{\ensuremath{{}^\circ}}}\end{align*}

The calculator gives approximate values of the trigonometric ratios. \ You
must look at your question to check whether angles are in degrees or radians and ensure the calculator is first set in the right \emph{mode}.
\ Remember all questions where degrees are to be used will give angles marked with a $\mbox{{\ensuremath{{}^\circ}}}$ symbol. \ You could assume though that when you
are dealing with a static problem (i.e. one where a triangle is given and sides and angles are required) then the angle will be measured in degrees. 

When you \ are required to find a side you select the trigonometric ratio and either multiply or divide.
\ When you are required to find an angle you select the appropriate trigonometry ratio and use \emph{shift}
with sin, cos or tan to find $\sin ^{ -1}$, $\cos ^{ -1}$ or $\tan ^{ -1}$ because $\sin ^{ -1}$ is said "the angle whose sine is" etc. 

\subsection{Applications.}
The exercises on pp 489-492 start with some basic questions (1-16) and you should check you can do them. \ Questions
follow (17-28) that test your understanding of the relationship between Pythagoras theorem and the trigonometry of a right triangle. \ There
are four questions (29-32) that give two facts about a right triangle and ask you to find the other three facts. \ Questions
33-59 should be the focus of your attention. \ Trigonometry is a practical topic and there is a good selection
of practical questions here. 

\subsubsection{Example 1}
The height of a steep cliff is to be measured from a point on the opposite side of the river. \ \ \ The
following diagram shows the measurements taken. \ Estimate the height of the cliff. 


%TCIMACRO{\TeXButton{Start Two Columns}{\columnsep =30pt
% \begin {multicols}{2}}}%
%BeginExpansion
\columnsep =30pt
\begin {multicols}{2}
%EndExpansion
 

   
\setlength\fboxrule{0in}\setlength\fboxsep{0.2in}\fcolorbox[HTML]{000000}{FFFFFF}{\includegraphics[ width=3.045in, height=2.9213in,]{L4SZ281H}
}



\begin{align*}\frac{d}{58.2} &  = & \tan  50.0 \mbox{{\ensuremath{{}^\circ}}} \\
d &  = & 58.2 \times \tan  50.0 \mbox{{\ensuremath{{}^\circ}}} \\
\frac{h}{d} &  = & \tan  76.3 \mbox{{\ensuremath{{}^\circ}}} \\
h &  = & d \times \tan  76.3 \mbox{{\ensuremath{{}^\circ}}} \\
 &  = & 58.2 \times \tan  50.0 \mbox{{\ensuremath{{}^\circ}}} \times \tan  76.3 \mbox{{\ensuremath{{}^\circ}}} \\
 &  \approx  & 284.526397 \\
 &  \approx  & 284.5 \mbox{m}\end{align*}


%TCIMACRO{\TeXButton{End Two Columns}{\end {multicols}}}%
%BeginExpansion
\end {multicols}
%EndExpansion


\subsubsection{Example 2}
To estimate the height of a mountain above a level plane the angle of elevation of the top of the mountain is measured to be $30 \mbox{{\ensuremath{{}^\circ}}}$. \ $600 \mbox{m}$ closer to the mountain across the plane it is found that the angle of elevation
is $36 \mbox{{\ensuremath{{}^\circ}}}$. \ Estimate the height of the mountain. 

   
\setlength\fboxrule{0in}\setlength\fboxsep{0.2in}\fcolorbox[HTML]{000000}{FFFFFF}{\includegraphics[ width=3.8649in, height=1.8412in,]{L4SZ281I}
}


\begin{equation*}\frac{h}{x} =\tan  36 \mbox{{\ensuremath{{}^\circ}}}\text{and}\frac{h}{x +600} =\tan  30 \mbox{{\ensuremath{{}^\circ}}}
\end{equation*}

We want $h$ so we eliminate $x$ between these two equations
\begin{align*}x &  = & \frac{h}{\tan  36 \mbox{{\ensuremath{{}^\circ}}}}\text{and}x +600 =\frac{h}{\tan  30 \mbox{{\ensuremath{{}^\circ}}}} \\
\frac{h}{\tan  30 \mbox{{\ensuremath{{}^\circ}}}} &  = & \frac{h}{\tan  36 \mbox{{\ensuremath{{}^\circ}}}} +600 \\
\frac{h}{\tan  30 \mbox{{\ensuremath{{}^\circ}}}} -\frac{h}{\tan  36 \mbox{{\ensuremath{{}^\circ}}}} &  = & 600 \\
h \left (\frac{1}{\tan  30 \mbox{{\ensuremath{{}^\circ}}}} -\frac{1}{\tan  36 \mbox{{\ensuremath{{}^\circ}}}}\right ) &  = & 600 \\
h \genfrac{(}{)}{}{}{\tan  36 \mbox{{\ensuremath{{}^\circ}}} -\tan  30 \mbox{{\ensuremath{{}^\circ}}}}{\tan  30 \mbox{{\ensuremath{{}^\circ}}} \tan  36 \mbox{{\ensuremath{{}^\circ}}}} &  = & 600 \\
h &  = & 600 \times \frac{\tan  30 \mbox{{\ensuremath{{}^\circ}}} \tan  36 \mbox{{\ensuremath{{}^\circ}}}}{\tan  36 \mbox{{\ensuremath{{}^\circ}}} -\tan  30 \mbox{{\ensuremath{{}^\circ}}}} \\
 &  \approx  & 600 \times 2.811603815 \\
 &  \approx  & 1687 \mbox{m}\end{align*}

\subsubsection{Example 3}
Now use the same method to show that for 

   
\setlength\fboxrule{0in}\setlength\fboxsep{0.2in}\fcolorbox[HTML]{000000}{FFFFFF}{\includegraphics[ width=3.8856in, height=1.8619in,]{L4SZ281J}
}
\begin{equation*}h =d \frac{\tan  \alpha  \tan  \beta }{\tan  \beta  -\tan  \alpha }
\end{equation*}

Let $x$ be the distance shown on the diagram.
\begin{align*}\frac{h}{x} &  = & \tan  \beta \text{and}\frac{h}{x +d} =\tan  \alpha  \\
\text{So}x &  = & \frac{h}{\tan  \beta }\text{and}x +d =\frac{h}{\tan  \alpha } \\
\text{Then}\frac{h}{\tan  \beta } +d &  = & \frac{h}{\tan  \alpha } \\
\frac{h}{\tan  \alpha } -\frac{h}{\tan  \beta } &  = & d \\
h \left (\frac{1}{\tan  \alpha } -\frac{1}{\tan  \beta }\right ) &  = & d \\
h \genfrac{(}{)}{}{}{\tan  \beta  -\tan  \alpha }{\tan  \alpha  \tan  \beta } &  = & d \\
h &  = & d \frac{\tan  \alpha  \tan  \beta }{\tan  \beta  -\tan  \alpha }\end{align*}

\subsection{Exercises}
The following exercises from pp 489-492 have been covered in this section: 

Find the exact value of $\sin  \theta $, $\cos  \theta $ and $\tan  \theta $ of the angle $\theta $ in the triangle. 


\begin{description}
\item [1.]   
%TCIMACRO{\TeXButton{Start Two Columns}{\columnsep =30pt
% \begin {multicols}{2}}}%
%BeginExpansion
\columnsep =30pt
\begin {multicols}{2}
%EndExpansion
    
\setlength\fboxrule{0in}\setlength\fboxsep{0.2in}\fcolorbox[HTML]{000000}{FFFFFF}{\includegraphics[ width=1.1632in, height=0.9548in,]{L4SZ281K}
}


\item [3.]    
\setlength\fboxrule{0in}\setlength\fboxsep{0.2in}\fcolorbox[HTML]{000000}{FFFFFF}{\includegraphics[ width=2.8003in, height=1.2756in,]{L4SZ281L}
}
%TCIMACRO{\TeXButton{End Two Columns}{\end {multicols}}}%
%BeginExpansion
\end {multicols}
%EndExpansion
 

\item [5.]
\setlength\fboxrule{0in}\setlength\fboxsep{0.2in}\fcolorbox[HTML]{000000}{FFFFFF}{\includegraphics[ width=1.9562in, height=1.3932in,]{L4SZ281M}
}
\end{description}

Find the side labelled x. 


\begin{description}
\item [10.]   
%TCIMACRO{\TeXButton{Start Two Columns}{\columnsep =30pt
% \begin {multicols}{2}}}%
%BeginExpansion
\columnsep =30pt
\begin {multicols}{2}
%EndExpansion
    
\setlength\fboxrule{0in}\setlength\fboxsep{0.2in}\fcolorbox[HTML]{000000}{FFFFFF}{\includegraphics[ width=2.1698in, height=1.2254in,]{L4SZ281N}
}


\item [11.]    
\setlength\fboxrule{0in}\setlength\fboxsep{0.2in}\fcolorbox[HTML]{000000}{FFFFFF}{\includegraphics[ width=1.7348in, height=1.2462in,]{L4SZ281O}
}
%TCIMACRO{\TeXButton{End Two Columns}{\end {multicols}}}%
%BeginExpansion
\end {multicols}
%EndExpansion
 

\item [13.]
\setlength\fboxrule{0in}\setlength\fboxsep{0.2in}\fcolorbox[HTML]{000000}{FFFFFF}{\includegraphics[ width=1.996in, height=1.2289in,]{L4SZ281P}
}
\end{description}

Solve the right-angled triangle 


\begin{description}
\item [29.]   
%TCIMACRO{\TeXButton{Start Two Columns}{\columnsep =30pt
% \begin {multicols}{2}}}%
%BeginExpansion
\columnsep =30pt
\begin {multicols}{2}
%EndExpansion
    
\setlength\fboxrule{0in}\setlength\fboxsep{0.2in}\fcolorbox[HTML]{000000}{FFFFFF}{\includegraphics[ width=2.0055in, height=1.4909in,]{L4SZ281Q}
}


\item [31.]    
\setlength\fboxrule{0in}\setlength\fboxsep{0.2in}\fcolorbox[HTML]{000000}{FFFFFF}{\includegraphics[ width=1.5843in, height=2.3065in,]{L4SZ281R}
}
%TCIMACRO{\TeXButton{End Two Columns}{\end {multicols}}}%
%BeginExpansion
\end {multicols}
%EndExpansion
 

\item [35.]
The angle of elevation to the top of the Empire State Building in New York is found to be $11 \mbox{{\ensuremath{{}^\circ}}}$ from the ground at a distance of $1 \mbox{mi}$ from the base of the building. \ Using this information,
find the height of the Empire State Building. 

\item [35.]
A laser beam is to be directed towards the centre of the moon but the beam strays $0.5 \mbox{{\ensuremath{{}^\circ}}}$ from its intended path. 

\item [(a)]
How far has the beam diverged from its assigned target when it reaches the moon? \ (The distance of the earth
to the moon is $240$ $000 \mbox{mi}\text{.}$) 

\item [(b)] The radius
of the moon is about $1000 \mbox{mi}$. \ Will the beam strike the moon? 

\item [39.]
A $20 \mbox{ft}$ ladder leans against a building so that the angle between the ground and the ladder is $72 \mbox{{\ensuremath{{}^\circ}}}$. \ How high does the ladder reach on the building?


\item [42.] A $96 \mbox{ft}$ tree casts a shadow that is $120 \mbox{ft}$ long. \ What is the angle of elevation of the sun?


\item [43.] a man is lying on the beach, flying a kite. \ He
holds the end of the kite string at ground level, and estimates that the angle of elevation of the kite to be $50 \mbox{{\ensuremath{{}^\circ}}}$. \ If the string is $450 \mbox{ft}$ long, how high is the kite above the ground? 

\item [45.]
A water tower is located $325 \mbox{ft}$ from a building. \ From a window in the building
it is observed that the angle of elevation to the top of the tower is $39 \mbox{{\ensuremath{{}^\circ}}}$ and the angle of depression to the bottom of the tower is $25 \mbox{{\ensuremath{{}^\circ}}}$. \ How tall is the tower? \ How
high is the window? 

\item [46.] An airplane is flying at an
elevation of $5150 \mbox{ft}$ directly above a straight highway. \ Two motorists
are driving cars on the highway on opposite sides of the plane and the angle of depression to one car is $35 \mbox{{\ensuremath{{}^\circ}}}$ and to the other is $52 \mbox{{\ensuremath{{}^\circ}}}$. \ How far apart are the cars? 

\item [49.]
To estimate the height of a mountain above a level plain, the angle of elevation to the top of the mountain is measured to be $32 \mbox{{\ensuremath{{}^\circ}}}$. \ One thousand feet closed to the mountain along
the plain, it is found that the angle of elevation is $35 \mbox{{\ensuremath{{}^\circ}}}$. \ Estimate the height of the mountain. \end{description}

 

\section{3.3 The Trigonometric Functions of Angles}
Readings pp 493-500 

In section 8.2 the angles were between $0 \mbox{{\ensuremath{{}^\circ}}}$ and $90 \mbox{{\ensuremath{{}^\circ}}}$. \ In this section the angles can take any value.
\ Initially we consider angles between $0 \mbox{{\ensuremath{{}^\circ}}}$ and $360 \mbox{{\ensuremath{{}^\circ}}}$ and relate these to the previous chapter where $t$ varied between $0$ and $2 \pi $. \ We remind you that angles are measured anticlockwise from the positive $x$-axis. 

If the point $P (x ,y)$ is in the first quadrant, $\theta $ is the angle between $\vec{OP}$ and the positive $x$-axis and we complete the right triangle then we have created the situation in section 8.2.  
%TCIMACRO{\TeXButton{Start Two Columns}{\columnsep =30pt
% \begin {multicols}{2}}}%
%BeginExpansion
\columnsep =30pt
\begin {multicols}{2}
%EndExpansion
 

   
\setlength\fboxrule{0in}\setlength\fboxsep{0.2in}\fcolorbox[HTML]{000000}{FFFFFF}{\includegraphics[ width=3.774in, height=2.3454in,]{L4SZ281S}
}


Let the hypotenuse be $r$ then
\begin{equation*}r =\sqrt{x^{2} +y^{2}}
\end{equation*}


%TCIMACRO{\TeXButton{End Two Columns}{\end {multicols}}}%
%BeginExpansion
\end {multicols}
%EndExpansion


Therefore
\begin{equation*}\sin  \theta  =\frac{y}{r}\text{}\cos  \theta  =\frac{x}{r}\text{and}\tan  \theta  =\frac{y}{x}
\end{equation*}

We now let $\theta $ be any angle and define sine, cosine and tangent in the same way. \ For instance
if $P (x ,y)$ is in the second quadrant: 

   
\setlength\fboxrule{0in}\setlength\fboxsep{0.2in}\fcolorbox[HTML]{000000}{FFFFFF}{\includegraphics[ width=3.9868in, height=1.8161in,]{L4SZ281T}
}


$\sin  \theta  =\frac{y}{r}$ because $y$ is positive $\sin  \theta $ will be positive. ($r$ is positive by convention.) 

$\cos  \theta  =\frac{x}{r}$ because $x$ is negative $\cos  \theta $ will be negative. 

$\tan  \theta  =\frac{y}{x}$ because $y$ is positive and $x$ is negative $\tan  \theta $ will be negative. 

This pattern can be extended to quadrants 3 and 4. \ The
mnemonic mentioned in chapter 2 (\textbf{A}ll \textbf{S}tudents \textbf{T}ake \textbf{C}alculus) might help you remember which one is positive
although you can always work it out if you need to. 

\subsection{Relating Angles between 0$\mbox{{\ensuremath{{}^\circ}}}$ and 360$\mbox{{\ensuremath{{}^\circ}}}$ to an Equivalent Angle between 0$\mbox{{\ensuremath{{}^\circ}}}$ and 90$\mbox{{\ensuremath{{}^\circ}}}$}
By constructing $\bar{PQ}$ perpendicular to the $x$-axis a triangle is formed and this can be transformed into a congruent triangle in the first quadrant. \ This
is particularly important if an exact value is required for one of the special angles. 

\subsubsection{Example 1}
Given $\sin  60 \mbox{{\ensuremath{{}^\circ}}} =\frac{\sqrt{3}}{2}$ it can be shown $\sin  120 \mbox{{\ensuremath{{}^\circ}}} =\frac{\sqrt{3}}{2}$, $\sin  240 \mbox{{\ensuremath{{}^\circ}}} = -\frac{\sqrt{3}}{2}$ and $\sin  300 \mbox{{\ensuremath{{}^\circ}}} = -\frac{\sqrt{3}}{2}$ 

\subsubsection{Example 2}
Given $\cos  45 \mbox{{\ensuremath{{}^\circ}}} =\frac{\sqrt{2}}{2}$ it can be shown $\cos  135 \mbox{{\ensuremath{{}^\circ}}} = -\frac{\sqrt{2}}{2}$, $\cos  225 \mbox{{\ensuremath{{}^\circ}}} = -\frac{\sqrt{2}}{2}$ and $\cos  315 \mbox{{\ensuremath{{}^\circ}}} =\frac{\sqrt{2}}{2}$ 

This can be continued beyond $360 \mbox{{\ensuremath{{}^\circ}}}$. 

\subsubsection{Example 3}
Given $\tan  30 \mbox{{\ensuremath{{}^\circ}}} =\frac{\sqrt{3}}{3}$ it can be shown $\tan  390 \mbox{{\ensuremath{{}^\circ}}} =\frac{\sqrt{3}}{3}$, $\tan  510 \mbox{{\ensuremath{{}^\circ}}} = -\frac{\sqrt{3}}{3}$, $\tan  570 \mbox{{\ensuremath{{}^\circ}}} =\frac{\sqrt{3}}{3}$ and $\tan  690 \mbox{{\ensuremath{{}^\circ}}} = -\frac{\sqrt{3}}{3}$ 

This also holds for angles measured in a clockwise direction 

\subsubsection{Example 4}
Given $\tan  45 \mbox{{\ensuremath{{}^\circ}}} =1$ it can be shown $\tan  -45 \mbox{{\ensuremath{{}^\circ}}} = -1$, $\tan  -135 \mbox{{\ensuremath{{}^\circ}}} = -1$, $\tan  -225 \mbox{{\ensuremath{{}^\circ}}} = -1$ and $\tan  -315 \mbox{{\ensuremath{{}^\circ}}} =1$ 

The textbook refers to this as finding the \emph{reference angle}. \ We
will avoid this additional definition, however the procedure of finding the congruent triangle in the first quadrant must be understood. \ Although
the same logic can be applied to \emph{any} angle in that a congruent triangle can be constructed in the first quadrant we will only bother
to do this for the special angles. \ For all other angles the calculator gives an approximate value. \ You
should note therefore whether a special angle is given where an exact answer is expected or whether an approximate answer will suffice. 

\subsection{The Area of a Triangle}
The fundamental formula for the area of a triangle is
\begin{equation*}\text{Area} =\frac{1}{2} \times \text{base} \times \text{height}
\end{equation*}

Using the trigonometric functions the height can be replaced
and the formula becomes
\begin{equation*}\text{Area} =\frac{1}{2} \times \text{product of two sides} \times \text{sine of the included angle}
\end{equation*}

The formula is particularly easy to remember in symbolic form. \ Let
the triangle have vertices $A$, $B$ and $C$, so the the sides opposite these angles are $a$, $b$ and $c$ respectively. \ Then the area can be expressed symbolically as
\begin{equation*}\text{Area} =\frac{1}{2} a b \sin  C =\frac{1}{2} b c \sin  A =\frac{1}{2} a c \sin  B
\end{equation*}


%TCIMACRO{\TeXButton{Start Two Columns}{\columnsep =30pt
% \begin {multicols}{2}}}%
%BeginExpansion
\columnsep =30pt
\begin {multicols}{2}
%EndExpansion
 

   
\setlength\fboxrule{0in}\setlength\fboxsep{0.2in}\fcolorbox[HTML]{000000}{FFFFFF}{\includegraphics[ width=3.1012in, height=1.9415in,]{L4SZ281U}
}



\begin{align*}\sin  A &  = & \frac{h}{c}\text{so}h =c \sin  A \\
\text{Area} &  = & \frac{1}{2} \times \text{base} \times \text{height} \\
 &  = & \frac{1}{2} \times b \times c \sin  A =\frac{1}{2} b c \sin  A\end{align*}


%TCIMACRO{\TeXButton{End Two Columns}{\end {multicols}}}%
%BeginExpansion
\end {multicols}
%EndExpansion


It depends where you draw $h$ and which angle you choose to use as to which formula you finish up with. \ The key
point to remember is $b$ and $c$ are two sides and $A$ is the angle between them. \ The triangle above shows $A$ as an acute angle (between $0$ and $90 \mbox{{\ensuremath{{}^\circ}}}$). \ If the angle is obtuse (between $90 \mbox{{\ensuremath{{}^\circ}}}$ and $180 \mbox{{\ensuremath{{}^\circ}}}$) the formula still holds. 

   
\setlength\fboxrule{0in}\setlength\fboxsep{0.2in}\fcolorbox[HTML]{000000}{FFFFFF}{\includegraphics[ width=3.6573in, height=1.6769in,]{L4SZ281V}
}


The angle is in the second quadrant and $\sin  (180 -\theta ) =\sin  \theta $, so $\sin  (180 -\theta ) =\frac{h}{c}$ can be written as $\sin  \theta  =\frac{h}{c}$ or $h =c \sin  \theta $ or $h =c \sin  A$ where $A$ is obtuse. So the area is $\frac{1}{2} b c \sin  A$ 

\subsubsection{Example 5}
A triangle has two sides of $5 \mbox{cm}$ and $8 \mbox{cm}$ and the angle between them is $150 \mbox{{\ensuremath{{}^\circ}}}$. \ Find its area.
\begin{equation*}\text{Area} =\frac{1}{2} \times 5 \times 8 \times \sin  150 \mbox{{\ensuremath{{}^\circ}}}
\end{equation*}

It helps to remember that $\sin  150 =\sin  30 =\frac{1}{2}$
\begin{equation*}\text{Area} =\frac{1}{2} \times 5 \times 8 \times \frac{1}{2} =10 cm^{2}
\end{equation*}

\subsection{Exercises}
The following exercises from pp 501-502 have been covered in this section: 

Find the exact value of the trigonometric function



\begin{description}
\item [7.]   
%TCIMACRO{\TeXButton{Start Two Columns}{\columnsep =30pt
% \begin {multicols}{2}}}%
%BeginExpansion
\columnsep =30pt
\begin {multicols}{2}
%EndExpansion
 $\sin  150 \mbox{{\ensuremath{{}^\circ}}}$ 

\item [9.] $\cos  135 \mbox{{\ensuremath{{}^\circ}}}$ 
%TCIMACRO{\TeXButton{End Two Columns}{\end {multicols}}}%
%BeginExpansion
\end {multicols}
%EndExpansion
 

\item [11.]
%TCIMACRO{\TeXButton{Start Two Columns}{\columnsep =30pt
% \begin {multicols}{2}}}%
%BeginExpansion
\columnsep =30pt
\begin {multicols}{2}
%EndExpansion
 $\tan  \left ( -60 \mbox{{\ensuremath{{}^\circ}}}\right )$ 

\item [15.] $\cos  570 \mbox{{\ensuremath{{}^\circ}}}$ 
%TCIMACRO{\TeXButton{End Two Columns}{\end {multicols}}}%
%BeginExpansion
\end {multicols}
%EndExpansion
  

\item [17.]   
%TCIMACRO{\TeXButton{Start Two Columns}{\columnsep =30pt
% \begin {multicols}{2}}}%
%BeginExpansion
\columnsep =30pt
\begin {multicols}{2}
%EndExpansion
 $\tan  750 \mbox{{\ensuremath{{}^\circ}}}$ 

\item [19.] $\sin  \frac{2 \pi }{3}$ 
%TCIMACRO{\TeXButton{End Two Columns}{\end {multicols}}}%
%BeginExpansion
\end {multicols}
%EndExpansion
 

\item [21.]
%TCIMACRO{\TeXButton{Start Two Columns}{\columnsep =30pt
% \begin {multicols}{2}}}%
%BeginExpansion
\columnsep =30pt
\begin {multicols}{2}
%EndExpansion
 $\sin  \frac{3 \pi }{2}$ 

\item [23.] $\cos  \left ( -\frac{7 \pi }{3}\right )$ 
%TCIMACRO{\TeXButton{End Two Columns}{\end {multicols}}}%
%BeginExpansion
\end {multicols}
%EndExpansion
 

\item [29.]
$\tan  \frac{5 \pi }{2}$ \end{description}

Find the value of the trigonometric functions of $\theta $ from the information given 


\begin{description}
\item [41.] $\sin  \theta  =\frac{3}{5}\text{,}$ $\theta $ in quadrant II 

\item [43.]
$\tan  \theta  = -\frac{3}{4}\text{,}$ $\cos  \theta  >0$ 

\item [49.] If $\theta  =\pi /3\text{,}$ find the value of each expression. 

\item [(a)]
%TCIMACRO{\TeXButton{Start Two Columns}{\columnsep =30pt
% \begin {multicols}{2}}}%
%BeginExpansion
\columnsep =30pt
\begin {multicols}{2}
%EndExpansion
 $\sin  2 \theta \text{,}$ $2 \sin  \theta $ 

\item [(b)] $\sin  \frac{1}{2} \theta \text{,}$ $\frac{\sin  \theta }{2}$ 
%TCIMACRO{\TeXButton{End Two Columns}{\end {multicols}}}%
%BeginExpansion
\end {multicols}
%EndExpansion
 

\item [(c)]
$\sin ^{2} \theta \text{,}$ $\sin  \left (\theta ^{2}\right )$ 

\item [51.] Find the area of a triangle
with sides of length $7$ and $9$ and included angle $72 \mbox{{\ensuremath{{}^\circ}}}$. 

\item [53.]
A triangle has an area of $16 in^{2}\text{,}$ and two of the sides of the triangle have lengths $5 \mbox{in}$ and $7 \mbox{in}$. \ Find the angle included by these two sides. \end{description}


%TCIMACRO{\TeXButton{Start Two Columns}{\columnsep =30pt
% \begin {multicols}{2}}}%
%BeginExpansion
\columnsep =30pt
\begin {multicols}{2}
%EndExpansion
 


%TCIMACRO{\TeXButton{End Two Columns}{\end {multicols}}}%
%BeginExpansion
\end {multicols}
%EndExpansion
 

\section{3.4 The Sine Rule}
Readings pp 505-510 

In this section we use the Sine Rule to find the sides and angles in triangles without a right angle.
\ In the next section we use the Cosine Rule to find sides and angles in triangles also, so as you study these
two sections you need to learn which problems require the Sine Rule and which require the Cosine Rule. 

In section 3.3 we met the formula
for the area of a triangle. given two sides and the included angle $\left ( =\frac{1}{2} a b \sin  C\right )$. \ The Sine Rule and
Cosine Rule also require specific combinations of sides and angles. 

The easiest way to visualise the situations in which the two rules
are used is to use the labelling we met in section $3.3$. \ Let the triangle be $ \Delta A B C$ and let the sides be $a$, $b$ and $c$ where $a$ is opposite $\angle A$ etc. 

   
\setlength\fboxrule{0in}\setlength\fboxsep{0.2in}\fcolorbox[HTML]{000000}{FFFFFF}{\includegraphics[ width=3.1358in, height=2.4171in,]{L4SZ281W}
}



\begin{tabular}[c]{lll}\textbf{Sine Rule}  &  & \textbf{Cosine
Rule}  \\
 Given a side and the angle opposite the side  &  & \textbf{(a)}
Given two sides and the included angle  \\
 \textbf{(a)} If a second angle is
given the Sine Rule  &  & the Cosine Rule allows us to find the
side  \\
allows us the side opposite that angle  &  & opposite
the angle  \\
Given $A$, $a$ and $B$ use the Sine Rule to find $b\text{.}$  &  & Given $a$, $b$ and $C$ use the Cosine Rule to find $c\text{.}$  \\
 \textbf{(b)} If a second
angle is given and the side that  &  & \textbf{(b)} Given three
sides the Cosine Rule allows  \\
is not opposite that angle is required then first  &  & us
to find any angle.  \\
find the third angle then use (a).  &  & Given
$a$, $b$ and $c$ use the Cosine Rule to find $A$  \\
Given $A$, $a$ and $B$ where $c$ is required  &  & or $B$ or $C$.  \\
 1. $C =180 -(A +B)$  &  &  \\
2.
Use the Sine Rule to find $c$.  &  &  \\
\textbf{(c)} If a second side is given the Sine Rule  &  &  \\
allows
us to find the angle opposite that side.  &  &  \\
Given
$A$, $a$, and $b$ use the Sine Rule to find $B$.  &  &  \\
\end{tabular}

The textbook uses the term "The Law of Sines" whereas in these notes the term
"The Sine Rule will be used. \ Similarly we will use the term "The Cosine Rule" instead of the term "The Law of
Cosines". 

The Sine Rule states that in any triangle
\begin{equation*}\frac{\sin  A}{a} =\frac{\sin  B}{b} =\frac{\sin  C}{c}
\end{equation*}

or
\begin{equation*}\frac{a}{\sin  A} =\frac{b}{\sin  B} =\frac{c}{\sin  C}
\end{equation*}

\subsection{Proof of the Sine Rule}
The Sine Rule is easy to prove from the formula for the area of a triangle
\begin{equation*}\text{Area} =\frac{1}{2} b c \sin  A =\frac{1}{2} a c \sin  B =\frac{1}{2} a b \sin  C
\end{equation*}

Multiply right through by 2
\begin{equation*}b c \sin  A =a c \sin  B =a b \sin  C
\end{equation*}

Divide right through by $a b c$
\begin{equation*}\frac{\sin  A}{a} =\frac{\sin  B}{b} =\frac{\sin  C}{c}
\end{equation*}

Fractions can always be inverted as long as the same process is applied to each fraction.
\begin{equation*}\frac{a}{\sin  A} =\frac{b}{\sin  B} =\frac{c}{\sin  C}
\end{equation*}

\subsubsection{Example 1}
   
\setlength\fboxrule{0in}\setlength\fboxsep{0.2in}\fcolorbox[HTML]{000000}{FFFFFF}{\includegraphics[ width=2.6143in, height=2.0159in,]{L4SZ281X}
}
\\\relax (a) To find $c$
\begin{align*}\frac{c}{\sin  C} &  = & \frac{b}{\sin  B} \\
\frac{c}{\sin  43 \mbox{{\ensuremath{{}^\circ}}}} &  = & \frac{5}{\sin  35 \mbox{{\ensuremath{{}^\circ}}}} \\
c &  = & \frac{5 \sin  43 \mbox{{\ensuremath{{}^\circ}}}}{\sin  35 \mbox{{\ensuremath{{}^\circ}}}} \\
 &  \approx  & 5.945139277 \approx 5.9 \mbox{cm}\text{(1 dp)}\end{align*} \\\relax (b) To find $a$ 

(i) $A =180 \mbox{{\ensuremath{{}^\circ}}} -(35 \mbox{{\ensuremath{{}^\circ}}} +43 \mbox{{\ensuremath{{}^\circ}}}) =180 \mbox{{\ensuremath{{}^\circ}}} -78 \mbox{{\ensuremath{{}^\circ}}} =102 \mbox{{\ensuremath{{}^\circ}}}$ 

(ii) It is usually wise to go back to the original data (i.e. use $b$ and $B$ rather than $c$ and $C$).
\begin{align*}\frac{a}{\sin  A} &  = & \frac{b}{\sin  B} \\
\frac{a}{\sin  102 \mbox{{\ensuremath{{}^\circ}}}} &  = & \frac{5}{\sin  35 \mbox{{\ensuremath{{}^\circ}}}} \\
a &  = & \frac{5 \sin  102 \mbox{{\ensuremath{{}^\circ}}}}{\sin  35 \mbox{{\ensuremath{{}^\circ}}}} \\
 &  \approx  & 8.526741501 \approx 8.5 \mbox{cm}\text{(1 dp)}\end{align*}

Examples 1 and 2 pp 506-507 provide additional examples. 

These two calculations
illustrate the first two cases in which the Sine Rule is used. \ You will notice that the triangle has been completely
solved in the course of this example. \ We started with one side and two angles and we found the other two sides
and the other angle. 

The third case is not as straight forward. \ Given two sides and an
angle there could be no triangle formed, one triangle formed or two triangles formed depending on the length of the side opposite the given angle. \ this
is illustrated in the diagrams on p 507 in the textbook. \ You should develop an insight into the reasons why
this is so. \ Imagine the second side given is the boom of a crane and the angle given is the angle between the
boom and the ground. \ The side opposite the given angle is represented by the cable. It is clear that for certain
lengths of the cable the hook will not reach the ground. \ Then as the hook is lowered a point will be reached
when the hook just touches the ground.  
%TCIMACRO{\TeXButton{Start Two Columns}{\columnsep =30pt
% \begin {multicols}{2}}}%
%BeginExpansion
\columnsep =30pt
\begin {multicols}{2}
%EndExpansion
 

   
\setlength\fboxrule{0in}\setlength\fboxsep{0.2in}\fcolorbox[HTML]{000000}{FFFFFF}{\includegraphics[ width=2.9577in, height=2.2061in,]{L4SZ281Y}
}


It is no surprise that the length $a$ to create this situation is $a =b \sin  A\text{.}$ 

We are after all dealing with the Sine Rule which becomes the fundamental sine formula $\left (\sin  A =\frac{\text{opp}}{\text{hyp}} =\frac{a}{b}\right )$ when the triangle has a right angle. 
%TCIMACRO{\TeXButton{End Two Columns}{\end {multicols}}}%
%BeginExpansion
\end {multicols}
%EndExpansion


If the cable is held taut and is extended a little more it will touch the ground in two places (provided it is kept in the same plane).
\ As the cable is extended further the time will come where the cable is as long as boom. \ At
this point there is only one solution again, the one straight out in front of the crane). \ Further extensions
of the cable will produce only one solution (straight out in front of the crane) as the cable will theoretically reach behind the crane boom thus creating
a different triangle altogether. 

The two solutions case is often referred to as the ambiguous case. \ The
discussion above shows the range of values of $a$ that will give two solutions.
\begin{equation*}b \sin  A <a <b
\end{equation*}

\subsubsection{Example 2}
Given $a =30 \mbox{{\ensuremath{{}^\circ}}}$, $a =8$ and $b =7$ solve the triangle (i.e. find $B$, $C$ and $c$). 

   
\setlength\fboxrule{0in}\setlength\fboxsep{0.2in}\fcolorbox[HTML]{000000}{FFFFFF}{\includegraphics[ width=4.2739in, height=1.2202in,]{L4SZ281Z}
}


Because $8 >7$ this is the one solution case. 

1. Find $B$
\begin{align*}\frac{\sin  B}{b} &  = & \frac{\sin  A}{a} \\
\sin  B &  = & \frac{b \sin  A}{a} \\
 &  = & \frac{7 \sin  30 \mbox{{\ensuremath{{}^\circ}}}}{8} =0.4375 \\
B &  = & \sin ^{ -1} 0.4375 \approx 25.94 \mbox{{\ensuremath{{}^\circ}}}\end{align*}

2. Find $C$
\begin{align*}C &  = & 180 \mbox{{\ensuremath{{}^\circ}}} -(30 \mbox{{\ensuremath{{}^\circ}}} +25.94 \mbox{{\ensuremath{{}^\circ}}}) \\
 &  = & 124.06 \mbox{{\ensuremath{{}^\circ}}}\end{align*}

3. Find $c$
\begin{align*}\frac{c}{\sin  C} &  = & \frac{a}{\sin  A} \\
c &  = & \frac{a \sin  C}{c} \\
 &  = & \frac{7 \sin  124.06 \mbox{{\ensuremath{{}^\circ}}}}{\sin  30 \mbox{{\ensuremath{{}^\circ}}}} \\
 &  \approx  & 11.6\end{align*}

\subsubsection{Example 3}
Given $A =30 \mbox{{\ensuremath{{}^\circ}}}$, $a =6$ and $b =7$ solve the triangle. 

In this case $b \sin  A =3.5$ and $b =7$ so as $a$ lies between $3.5$ and $7$. \ This is the ambiguous case, therefore there are two solutions. 


%TCIMACRO{\TeXButton{Start Two Columns}{\columnsep =30pt
% \begin {multicols}{2}}}%
%BeginExpansion
\columnsep =30pt
\begin {multicols}{2}
%EndExpansion
 

   
\setlength\fboxrule{0in}\setlength\fboxsep{0.2in}\fcolorbox[HTML]{000000}{FFFFFF}{\includegraphics[ width=2.8885in, height=1.2825in,]{L4SZ2820}
}


   
\setlength\fboxrule{0in}\setlength\fboxsep{0.2in}\fcolorbox[HTML]{000000}{FFFFFF}{\includegraphics[ width=2.9308in, height=1.3015in,]{L4SZ2821}
}



%TCIMACRO{\TeXButton{End Two Columns}{\end {multicols}}}%
%BeginExpansion
\end {multicols}
%EndExpansion
 

It
is best to visualise these two solutions on the same diagram so that the isosceles triangle can help lead to the two results. 

   
\setlength\fboxrule{0in}\setlength\fboxsep{0.2in}\fcolorbox[HTML]{000000}{FFFFFF}{\includegraphics[ width=2.9585in, height=1.3136in,]{L4SZ2822}
}


 \textbf{First solution} (Proceed as before) 

1. Find $B$
\begin{align*}\frac{\sin  B}{b} &  = & \frac{\sin  A}{a} \\
\sin  B &  = & \frac{b \sin  A}{a} \\
 &  = & \frac{7 \sin  30 \mbox{{\ensuremath{{}^\circ}}}}{6} =0.58 \dot{3} \\
B &  = & \sin ^{ -1} 0.58 \dot{3} \approx 35.69 \mbox{{\ensuremath{{}^\circ}}}\end{align*}

2. Find $C$
\begin{align*}C &  = & 180 \mbox{{\ensuremath{{}^\circ}}} -(30 \mbox{{\ensuremath{{}^\circ}}} +35.69 \mbox{{\ensuremath{{}^\circ}}}) \\
 &  = & 114.31 \mbox{{\ensuremath{{}^\circ}}}\end{align*}

3. Find $c$
\begin{align*}\frac{c}{\sin  C} &  = & \frac{a}{\sin  A} \\
c &  = & \frac{a \sin  C}{\sin  A} \\
 &  = & \frac{6 \sin  114.31 \mbox{{\ensuremath{{}^\circ}}}}{\sin  30 \mbox{{\ensuremath{{}^\circ}}}} \\
 &  \approx  & 10.9\end{align*}

\textbf{Second solution} 

1. Find the second value of $B$ 

\begin{equation*}B =180 \mbox{{\ensuremath{{}^\circ}}} -35.69 \mbox{{\ensuremath{{}^\circ}}} =144.31 \mbox{{\ensuremath{{}^\circ}}}
\end{equation*}

2. Find $C$
\begin{equation*}C =180 \mbox{{\ensuremath{{}^\circ}}} -(30 \mbox{{\ensuremath{{}^\circ}}} +144.31 \mbox{{\ensuremath{{}^\circ}}}) =180 \mbox{{\ensuremath{{}^\circ}}} -174.31 \mbox{{\ensuremath{{}^\circ}}} =5.69 \mbox{{\ensuremath{{}^\circ}}}
\end{equation*}

(Or if you remember the rule that the exterior angle of a triangle is the sum of the two interior
opposite angles $C +30 \mbox{{\ensuremath{{}^\circ}}} =35.69 \mbox{{\ensuremath{{}^\circ}}}$ so $C =35.69 \mbox{{\ensuremath{{}^\circ}}} -30 \mbox{{\ensuremath{{}^\circ}}} =5.69 \mbox{{\ensuremath{{}^\circ}}}$) 

3. Find $c$
\begin{align*}\frac{c}{\sin  C} &  = & \frac{a}{\sin  A} \\
c &  = & \frac{a \sin  C}{\sin  A} \\
 &  = & \frac{6 \sin  5.69 \mbox{{\ensuremath{{}^\circ}}}}{\sin  30 \mbox{{\ensuremath{{}^\circ}}}} \\
 &  \approx  & 1.2\end{align*}

%\subsection{ }


\subsection{Exercises}
The following exercises from pp510-512 have been covered in this section: 

Use the Sine Rule to find side $x$ or angle $\theta $  
%TCIMACRO{\TeXButton{Start Two Columns}{\columnsep =30pt
% \begin {multicols}{2}}}%
%BeginExpansion
\columnsep =30pt
\begin {multicols}{2}
%EndExpansion
 


%TCIMACRO{\TeXButton{End Two Columns}{\end {multicols}}}%
%BeginExpansion
\end {multicols}
%EndExpansion



\begin{description}
\item [1.]   
%TCIMACRO{\TeXButton{Start Two Columns}{\columnsep =30pt
% \begin {multicols}{2}}}%
%BeginExpansion
\columnsep =30pt
\begin {multicols}{2}
%EndExpansion
    
\setlength\fboxrule{0in}\setlength\fboxsep{0.2in}\fcolorbox[HTML]{000000}{FFFFFF}{\includegraphics[ width=2.2373in, height=1.7902in,]{L4SZ2823}
}


\item [5]    
\setlength\fboxrule{0in}\setlength\fboxsep{0.2in}\fcolorbox[HTML]{000000}{FFFFFF}{\includegraphics[ width=2.3609in, height=1.241in,]{L4SZ2824}
}
%TCIMACRO{\TeXButton{End Two Columns}{\end {multicols}}}%
%BeginExpansion
\end {multicols}
%EndExpansion
 \end{description}

Solve the triangle using the Sine Rule. 


\begin{description}
\item [7.]    
\setlength\fboxrule{0in}\setlength\fboxsep{0.2in}\fcolorbox[HTML]{000000}{FFFFFF}{\includegraphics[ width=3.5198in, height=1.3863in,]{L4SZ2825}
}
\end{description}

Sketch each triangle and then solve the triangle using the Sine Rule. 


\begin{description}
\item [9.] $\angle A =50 \mbox{{\ensuremath{{}^\circ}}}\text{,}$ $\angle B =68 \mbox{{\ensuremath{{}^\circ}}}\text{,}$ $c =230$ 

\item [13.] $\angle B =29 \mbox{{\ensuremath{{}^\circ}}}\text{,}$ $\angle C =51 \mbox{{\ensuremath{{}^\circ}}}\text{,}$ $b =44$ \end{description}

Use the Sine Rule to solve for all possible triangles
that satisfy the given conditions. 


\begin{description}
\item [15.] $a =28\text{,}$ $b =15\text{,}$ $\angle A =110 \mbox{{\ensuremath{{}^\circ}}}$ 

\item [17.] $a =20\text{,}$ $c =45\text{,}$ $\angle A =125 \mbox{{\ensuremath{{}^\circ}}}$ 

\item [19.] $b =25\text{,}$ $c =30\text{,}$ $\angle B =25 \mbox{{\ensuremath{{}^\circ}}}$ 

\item [23.]   
%TCIMACRO{\TeXButton{Start Two Columns}{\columnsep =30pt
% \begin {multicols}{2}}}%
%BeginExpansion
\columnsep =30pt
\begin {multicols}{2}
%EndExpansion
 To find the distance across a river, a surveyor chooses points $A$ and $B$, which are $200 \mbox{ft}$ apart on one side of the river. \ She then chooses
a reference point $C$ on the opposite side of the river and finds that $\angle BAC \approx 82 \mbox{{\ensuremath{{}^\circ}}}$ and $\angle ABC \approx 52 \mbox{{\ensuremath{{}^\circ}}}\text{.}$ \ \ Find the approximate
distance from $A$ to $C$. 

\item    
\setlength\fboxrule{0in}\setlength\fboxsep{0.2in}\fcolorbox[HTML]{000000}{FFFFFF}{\includegraphics[ width=2.4059in, height=1.7988in,]{L4SZ2826}
}
%TCIMACRO{\TeXButton{End Two Columns}{\end {multicols}}}%
%BeginExpansion
\end {multicols}
%EndExpansion
 

\item [25.]
The path of a satellite circling the earth causes it to pass directly over two tracking stations $A$ and $B$, which are $50 \mbox{mi}$ apart. \ When the satellite is on one side of the
two stations, the angle of elevation at $A$ and $B$ are measured to be $87.0 \mbox{{\ensuremath{{}^\circ}}}$ and $84.2 \mbox{{\ensuremath{{}^\circ}}}$, respectively. 

\item [(a)]
Draw a diagram. 

\item [(b)] How far is the satellite from
station $A$? 

\item [(c)] How high is the satellite
above the ground? 

\item [27.]   
%TCIMACRO{\TeXButton{Start Two Columns}{\columnsep =30pt
% \begin {multicols}{2}}}%
%BeginExpansion
\columnsep =30pt
\begin {multicols}{2}
%EndExpansion
 A communication tower is located at the top of a steep hill. \ The angle of inclination
of the hill is $58 \mbox{{\ensuremath{{}^\circ}}}$. \ A guy wire is attached to the top of the tower
and to the ground, $100 \mbox{m}$ downhill from the base of the tower. \ The
angle between the slope of the hill and the guy wire is measured as $12 \mbox{{\ensuremath{{}^\circ}}}$. \ Find $A C$, the length of cable required for the guy wire. 

\item    
\setlength\fboxrule{0in}\setlength\fboxsep{0.2in}\fcolorbox[HTML]{000000}{FFFFFF}{\includegraphics[ width=2.0851in, height=2.3575in,]{L4SZ2827}
}
%TCIMACRO{\TeXButton{End Two Columns}{\end {multicols}}}%
%BeginExpansion
\end {multicols}
%EndExpansion
 

\item [31.]
%TCIMACRO{\TeXButton{Start Two Columns}{\columnsep =30pt
% \begin {multicols}{2}}}%
%BeginExpansion
\columnsep =30pt
\begin {multicols}{2}
%EndExpansion
 A water tower $30 \mbox{m}$ tall is located at the top of a hill. \ From
a distance of $120 \mbox{m}$ down the hill it is observer that the angle formed between the top and
the base of the tower is $8 \mbox{{\ensuremath{{}^\circ}}}$. \ Find $\angle ABC\text{,}$ the angle of inclination of the hill. 

\item    
\setlength\fboxrule{0in}\setlength\fboxsep{0.2in}\fcolorbox[HTML]{000000}{FFFFFF}{\includegraphics[ width=2.4829in, height=1.5281in,]{L4SZ2828}
}
%TCIMACRO{\TeXButton{End Two Columns}{\end {multicols}}}%
%BeginExpansion
\end {multicols}
%EndExpansion
 \end{description}

\section{Excel Exercise}


\subsection{The Sine Rule - Investigating the Ambiguous Case}
In the notes we discussed the ambiguous case and we looked at the triangle $ \Delta A B C$. \ Given $A$, $a$ and $b$ we found two critical values of $a$ that determined whether there would be $0$, $1$ or $2$ solutions for the triangle. \ The results are summarised in the following table. 

\qquad \qquad \qquad \qquad \qquad
\begin{tabular}[c]{|c|l|}\hline
$ <b \sin  A$  & $0$  \\
\hline
$ =b \sin  A$  & $1$ solution (right triangle)  \\
\hline
$b \sin  A <a <b$  & $2$ solutions  \\
\hline
$ =b$  & $1$ solution (one triangle is formed)  \\
\hline
$ >b$  & $1$ solution (second triangle invalid)  \\
\hline
\end{tabular}

Let $A =30 \mbox{{\ensuremath{{}^\circ}}}$ and let $b =7$ 

The task is to set out in a table the solutions for the triangle for the ambiguous case. \ (When
there are 2 solutions.) 


\begin{enumerate}
\item Find the two values of $a$ that define the limits when two solutions are obtained. 

\item Explore Excel
to check you know how to use the \emph{sine} function. 

\item In a column list values
of $a$ with increments of $0.1\text{.}$ 

\item Complete the first row of the table to give the two solutions.


\item Use \textbf{Fill Down} to complete the table. 

\item Inspect
the table to ensure you are satisfied you understand the pattern produced. \end{enumerate}


Look at the
values for $c$ and explain why the value at the top of the column is half the value at the bottom. 

Hints: 


\begin{enumerate}
\item Excel requires angles to be measured in radians so convert, (Use $A \times \pi  \div 180$). 

\item To find B use $\sin ^{ -1} \genfrac{(}{)}{}{}{b \sin  A}{a}$. \ This angle is in radians so must now be converted to degrees. 

\item To
find $\sin ^{ -1}$ use the function a$\sin \text{.}$ \ (This is short for $\arcsin $ which is what Excel and some textbooks use for $\sin ^{ -1}\text{.}$) 

\item To find $C$ use $180 -(A +B)$ 

\item To find $c$ convert $A$ and $C$ to radians ($ \times 180 \div \pi $) and use $c =\frac{a \sin  C}{\sin  A}$ \end{enumerate}


An example of the final result can be found below. 

   
\setlength\fboxrule{0.01in}\setlength\fboxsep{0.2in}\fcolorbox[HTML]{000000}{FFFFFF}{\includegraphics[ width=6.077in, height=5.9698in,]{L4SZ2829}
}



%TCIMACRO{\TeXButton{Start Two Columns}{\columnsep =30pt
% \begin {multicols}{2}}}%
%BeginExpansion
\columnsep =30pt
\begin {multicols}{2}
%EndExpansion
 


%TCIMACRO{\TeXButton{End Two Columns}{\end {multicols}}}%
%BeginExpansion
\end {multicols}
%EndExpansion
 

\section{3.5 The Cosine Rule}
Readings pp 513-517 

In this section we will state and prove the Cosine Rule (which is called "The Law of Cosines" in the
textbook). \ The proof is given here for completeness. \ You will not
be tested on your ability to reproduce it. \ The section will give examples where the Cosine Rule is used to solve
problems using the \emph{triangle of vectors} and we will include revision of \emph{bearings} and the use of trigonometry
in \emph{navigation}. 

In this course we have mentioned so far two formulae for the area of a triangle and many problems
allow us to use one of those two formulae. \ There is a third formula that is used when the three sides of the
triangle are given. \ In practical situations this is often the easiest and most likely data that has been collected
so this method might be the most useful of the three. \ The formula is named after the person who first derived
it. \ It is called \emph{Heron's Formula}. 

\subsection{Proof of The Cosine Rule}
\textbf{To prove:} For any triangle $ \Delta A B C\text{,}$ $a^{2} =b^{2} +c^{2} -2 b c \cos  A$  
%TCIMACRO{\TeXButton{Start Two Columns}{\columnsep =30pt
% \begin {multicols}{2}}}%
%BeginExpansion
\columnsep =30pt
\begin {multicols}{2}
%EndExpansion
 

\vspace{2cm}
\setlength\fboxrule{0in}\setlength\fboxsep{0.2in}\fcolorbox[HTML]{000000}{FFFFFF}{\includegraphics[ width=3.1021in, height=2.3549in,]{L4SZ282A}
}
By Pythagoras Theorem
\begin{align*}B C^{2} &  = & C D^{2} +D B^{2} \\
a^{2} &  = & \left (c -b \cos  A\right )^{2} +\left (b \sin  A\right )^{2} \\
 &  = & c^{2} -2 b c \cos  A +b^{2} \cos ^{2} A +b^{2} \sin ^{2} A \\
 &  = & c^{2} -2 b c \cos  A +b^{2} \left (\cos ^{2} A +\sin ^{2} A\right ) \\
 &  = & c^{2} -2 b c \cos  A +b^{2}\text{as}\cos ^{2} A +\sin ^{2} A =1\text{\ }\end{align*} 
%TCIMACRO{\TeXButton{End Two Columns}{\end {multicols}}}%
%BeginExpansion
\end {multicols}
%EndExpansion
 

This is usually written
\begin{equation*}a^{2} =b^{2} +c^{2} -2 b c \cos  A
\end{equation*}

The diagram has been drawn to simplify the way the proof unfolds. \ You
will see that by placing the vertex $A$ at the origin the side $a$ is found in terms of $b$, $c$, and $A$. \ The proof would have been the same had $A$ and $B$ been as shown and $C$ placed in the second quadrant. \ (Thus producing a triangle with an obtuse angle at
$A$.) \ This rule is symmetrical. \ You need
to be given two sides and the included angle ($b$, $c$ and $A$) \ and the formula allows you to calculate $a$. \ Most textbooks will therefore show you three equivalent formulae
\begin{align*}a^{2} &  = & b^{2} +c^{2} -2 b c \cos  A \\
b^{2} &  = & c^{2} +a^{2} -2 c a \cos  B \\
c^{2} &  = & a^{2} +b^{2} -2 a b \cos  C\end{align*}

In this course you will only be given one formula and you will have to know it is a reminder to
lay out your solution in this way. \ In particular to remind you where to put the plus and minus signs. 

\subsubsection{Example 1}
Given $a =5$, $b =6$ and $C =50 \mbox{{\ensuremath{{}^\circ}}}$, find $c$.
\begin{align*}c^{2} &  = & a^{2} +b^{2} -2 a b \cos  C \\
 &  = & 5^{2} +6^{2} -2 \times 5 \times 6 \times \cos  50 \mbox{{\ensuremath{{}^\circ}}} \\
 &  \approx  & 22.43274342 \\
c &  \approx  & \sqrt{22.43274342} \\
 &  \approx  & 4.736321718 \approx 4.7 \left (1\text{dp}\right )\end{align*}

\subsubsection{Example 2}
Given $a =5$, $b =6$ and $C =130 \mbox{{\ensuremath{{}^\circ}}}$, find $c$.
\begin{align*}c^{2} &  = & a^{2} +b^{2} -2 a b \cos  C \\
 &  = & 5^{2} +6^{2} -2 \times 5 \times 6 \times \cos  130 \mbox{{\ensuremath{{}^\circ}}} \\
 &  \approx  & 99.56725658 \\
c &  \approx  & \sqrt{99.56725658} \\
 &  \approx  & 9.97833937 \approx 10.0 \left (1\text{dp}\right )\end{align*}

These two examples show that when the two sides and the included angle are given the third side
(opposite the given angle) can be found. \ This is more useful when a practical example is given and example 1
p 513-514 talks about a surveyor using the Cosine Rule to measure the length of a tunnel. \ However when the problem
is analysed it is still just what we have covered in example 1 above. 

\subsection{To Find an Angle Given Three Sides}
The Cosine Rule states $a^{2} =b^{2} +c^{2} -2 b c \cos  A\text{.}$ \ So if $a$, $b$, and $c$ are given $A$ can be calculated. \ The formula can be rearranged as follows
\begin{align*}2 b c \cos  A &  = & b^{2} +c^{2} -a^{2} \\
\cos  A &  = & \frac{b^{2} +c^{2} -a^{2}}{2 b c}\end{align*}


\begin{enumerate}
\item Again to use this formula you must appreciate the symmetry and the pattern of the numbers otherwise you are
just blindly substituting in a formula without understanding. \ Notice the fact that the minus sign is in front
of the $a^{2}$ term and a is opposite the angle we are finding. 

\item Notice the $b$ and $c$ are the sides that include the angle we are finding and these sides appear in both the numerator and denominator. \ Because
of the symmetry of the result we can write
\begin{align*}\cos  A &  = & \frac{b^{2} +c^{2} -a^{2}}{2 b c} \\
\cos  B &  = & \frac{c^{2} +a^{2} -b^{2}}{2 c a} \\
\cos  C &  = & \frac{a^{2} +b^{2} -c^{2}}{2 a b}\end{align*}\end{enumerate}


In this course you will be given one formula which
you should use by following the pattern it produces. 

\subsubsection{Example 3}
Given $a =5$, $b =6$ and $c =9$, find $A$. 

Because $a$ is the smallest side $A$ will most certainly be an acute angle.
\begin{align*}\cos  A &  = & \frac{b^{2} +c^{2} -a^{2}}{2 b c} \\
 &  = & \frac{6^{2} +9^{2} -5^{2}}{2 \times 6 \times 9} \\
 &  \approx  & 0.851851851 \\
A &  \approx  & \cos ^{ -1} \left (0.851851851\right ) \\
 &  \approx  & 31.6 \mbox{{\ensuremath{{}^\circ}}}\end{align*}

\subsubsection{Example 4}
For example 3 find $C$.
\begin{align*}\cos  C &  = & \frac{a^{2} +b^{2} -c^{2}}{2 a b} \\
 &  = & \frac{6^{2} +5^{2} -9^{2}}{2 \times 6 \times 5} \\
 &  \approx  &  -0. \dot{3} \\
C &  \approx  & \cos ^{ -1} \left ( -0. \dot{3}\right ) \\
 &  \approx  & 109.4712206 \\
 &  \approx  & 109.5 \mbox{{\ensuremath{{}^\circ}}}\end{align*}

\subsection{Navigation}
In a navigation question the direction is often given as a bearing. \ Thus $N 40 \mbox{{\ensuremath{{}^\circ}}} W$ is said "North $40 \mbox{{\ensuremath{{}^\circ}}}$ West" or "$40 \mbox{{\ensuremath{{}^\circ}}}$ West of North", and means start facing North and rotate $40 \mbox{{\ensuremath{{}^\circ}}}$ in a Westerly direction. \ A vector in this direction
is represented by an arrow. 

   
\setlength\fboxrule{0in}\setlength\fboxsep{0.2in}\fcolorbox[HTML]{000000}{FFFFFF}{\includegraphics[ width=2.3817in, height=2.4284in,]{L4SZ282B}
}


A journey therefore can comprise a sequence of direction changes and potential speed changes. \ A
trap in these problems is to confuse distances and speeds. \ You must read your question carefully and make sure
that every vector on the diagram has been correctly converted so that they are all either distances or speeds. 

\subsubsection{Example}
A fisherman leaves his home port and heads in a direction $N 70 \mbox{{\ensuremath{{}^\circ}}} W$. \ He travels $48 \mbox{km}$ to reach his first fishing spot. \ The following
day he travels in a direction $N 10 \mbox{{\ensuremath{{}^\circ}}} E$ at $8 \mbox{km}$/$\mbox{h}$ for $10$ hours to reach his second fishing spot. 


\begin{description}
\item [(a)] How far is he from his home port when he arrives at his second
fishing spot? 

\item [(b)] What is the bearing of the home
port from his second fishing spot? \end{description}

The first journey is in $\mbox{km}$ so the second journey must be in $\mbox{km}$ too so that we can draw a triangle of vectors. \ $8 \mbox{km}$/$\mbox{h}$ for $10$ hours is $80 \mbox{km}$. \ Assume both journeys are in straight lines. 

\textbf{Hints: } 


\begin{enumerate}
\item [I] Make a reasonable sized diagram so that the NSEW axes can be placed at each
vertex of the triangle of vectors. 

\item [II] Don't forget about corresponding
angles and alternate angles because the NSEW axes create parallel lines. \end{enumerate}



%TCIMACRO{\TeXButton{Start Two Columns}{\columnsep =30pt
% \begin {multicols}{2}}}%
%BeginExpansion
\columnsep =30pt
\begin {multicols}{2}
%EndExpansion
    
\setlength\fboxrule{0in}\setlength\fboxsep{0.2in}\fcolorbox[HTML]{000000}{FFFFFF}{\includegraphics[ width=3.4817in, height=5.3082in,]{L4SZ282C}
}



\begin{enumerate}
\item Sketch the situation. 

\item Use your knowledge of parallel
lines to show the angle (between the two journeys) at $B$ is $100 \mbox{{\ensuremath{{}^\circ}}}$. 

\item (a) You have two sides ($48$ and $80$) and the included angle ($100 \mbox{{\ensuremath{{}^\circ}}}$) so the\ Cosine Rule can be used to find $A C$ the journey back to port.
\begin{align*}b^{2} &  = & c^{2} +a^{2} -2 c a \cos  B \\
 &  = & 48^{2} +80^{2} -2 \times 48 \times 80 \times \cos  100 \mbox{{\ensuremath{{}^\circ}}} \\
 &  \approx  & 10037.618 \\
b &  \approx  & 100.1879135 \approx 100 \mbox{km}\end{align*}

\item (b) To find the bearing you must first find another angle
in the triangle. \ Find $C$. \ (Use the Sine Rule or Cosine Rule - both should work.)
\begin{align*}\frac{\sin  C}{c} &  = & \frac{\sin  B}{b} \\
\sin  C &  = & \frac{c \sin  B}{b} \\
 &  \approx  & \frac{48 \sin  100 \mbox{{\ensuremath{{}^\circ}}}}{100} \\
 &  \approx  & 0.472707721 \\
C &  \approx  & \sin ^{ -1} 0.472707721 \\
 &  \approx  & 28.21020463 \mbox{{\ensuremath{{}^\circ}}} \approx 28 \mbox{{\ensuremath{{}^\circ}}}\end{align*}\end{enumerate}



%TCIMACRO{\TeXButton{End Two Columns}{\end {multicols}}}%
%BeginExpansion
\end {multicols}
%EndExpansion


Put the $28 \mbox{{\ensuremath{{}^\circ}}}$ on the diagram and show that the angle between the vertical and the journey back to port
is $18 \mbox{{\ensuremath{{}^\circ}}}$. \ Bearing from South is $S 18 \mbox{{\ensuremath{{}^\circ}}} E\text{.}$ \ Bearing from North is $N 162 \mbox{{\ensuremath{{}^\circ}}} E$ 

\subsection{The Area of \ Triangle using Heron's Formula}
In this section we show you Heron's Formula to find the area of a triangle given the three sides. \ We
will then have three formulae that you could use and each requires different facts to be given. 


\begin{tabular}[c]{|l|l|}\hline
\textbf{Given}
& \textbf{Area}  \\
\hline
Base and height
& $\frac{1}{2} \times $ base $ \times $ height  \\
\hline
2 sides and the included angle
($a$,$b$ and $C$)  & $\frac{1}{2} a b \sin  C$  \\
\hline
3 sides ($a$, $b$ and $c$)  & $\sqrt{s \left (s -a\right ) \left (s -b\right ) \left (s -c\right )}$ where $s =\frac{1}{2} \left (a +b +c\right )$  \\
\hline
\end{tabular}

$s$ is called the \emph{semiperimeter}. 

This formula can be proved and the proof is in the
textbook pp 516-517. \ We will not discuss this proof in this course. \ Heron's
formula is very useful because you are more likely in practice to be required to find the area of a triangle whose sides are given than being given the
base and height or two sides and the included angle. 

\subsubsection{Example}
Given a triangle whose sides are $9 \mbox{cm}$, $10 \mbox{cm}$, and $11 \mbox{cm}$, find its area.
\begin{align*}s &  = & \frac{1}{2} \left (9 +10 +11\right ) \\
 &  = & 15 \\
\text{Area} &  = & \sqrt{15 (15 -9) (15 -10) (15 -11)} \\
 &  = & \sqrt{15 \times 6 \times 5 \times 4} \\
 &  \approx  & 42.42640687 \approx 42 cm^{2}\end{align*}

\subsection{Exercises}
The following exercises from pp 518-521 have been covered in this section: 


\begin{description}
\item [1.]   
%TCIMACRO{\TeXButton{Start Two Columns}{\columnsep =30pt
% \begin {multicols}{2}}}%
%BeginExpansion
\columnsep =30pt
\begin {multicols}{2}
%EndExpansion
 Use the Cosine Rule to find $x$ 

\item    
\setlength\fboxrule{0in}\setlength\fboxsep{0.2in}\fcolorbox[HTML]{000000}{FFFFFF}{\includegraphics[ width=2.8236in, height=1.3007in,]{L4SZ282D}
}
\vspace*{1.5cm} 

\item [5.]
Use the Cosine Rule to find $\theta $ 

\item    
\setlength\fboxrule{0in}\setlength\fboxsep{0.2in}\fcolorbox[HTML]{000000}{FFFFFF}{\includegraphics[ width=1.2747in, height=1.7262in,]{L4SZ282E}
}
%TCIMACRO{\TeXButton{End Two Columns}{\end {multicols}}}%
%BeginExpansion
\end {multicols}
%EndExpansion
 \end{description}

Solve the triangle ABC for questions 9, 13, and 15. 


\begin{description}
\item [9.]    
\setlength\fboxrule{0in}\setlength\fboxsep{0.2in}\fcolorbox[HTML]{000000}{FFFFFF}{\includegraphics[ width=2.693in, height=0.9798in,]{L4SZ282F}
}


\item [13.] $a =20\text{,}$ $b =25\text{,}$ $c =22$ 

\item [15.] $b =125\text{,}$ $c =162\text{,}$ $\angle B =40 \mbox{{\ensuremath{{}^\circ}}}$ \end{description}

For questions 19 and 23 use either the Sine Rule or Cosine
Rule as appropriate. 


\begin{description}
\item [19.]   
%TCIMACRO{\TeXButton{Start Two Columns}{\columnsep =30pt
% \begin {multicols}{2}}}%
%BeginExpansion
\columnsep =30pt
\begin {multicols}{2}
%EndExpansion
    
\setlength\fboxrule{0in}\setlength\fboxsep{0.2in}\fcolorbox[HTML]{000000}{FFFFFF}{\includegraphics[ width=2.5166in, height=1.1234in,]{L4SZ282G}
}


\item [23.]    
\setlength\fboxrule{0in}\setlength\fboxsep{0.2in}\fcolorbox[HTML]{000000}{FFFFFF}{\includegraphics[ width=2.412in, height=1.3309in,]{L4SZ282H}
}
%TCIMACRO{\TeXButton{End Two Columns}{\end {multicols}}}%
%BeginExpansion
\end {multicols}
%EndExpansion
 

\item [27.]
%TCIMACRO{\TeXButton{Start Two Columns}{\columnsep =30pt
% \begin {multicols}{2}}}%
%BeginExpansion
\columnsep =30pt
\begin {multicols}{2}
%EndExpansion
 To find the distance across a small lake, a surveyor has taken the measurements shown. \ Find
the distance across the lake using this information. 

\item    
\setlength\fboxrule{0in}\setlength\fboxsep{0.2in}\fcolorbox[HTML]{000000}{FFFFFF}{\includegraphics[ width=2.7665in, height=1.5965in,]{L4SZ282I}
}
%TCIMACRO{\TeXButton{End Two Columns}{\end {multicols}}}%
%BeginExpansion
\end {multicols}
%EndExpansion
 

\item [29.]
Two straight roads diverge at an angle of $65 \mbox{{\ensuremath{{}^\circ}}}$. \ Two cars leave the intersection at $2.00$ P.M., one traveling at $50 mi/\mbox{h}$ and the other at $30 mi/\mbox{h}$. \ How far apart are the cars at $2.30$ P.M.? 

\item [31.] A pilot flies
in a straight path for $1 \mbox{h}\; 30 \mbox{min}$. \ She then makes a course correction, heading $10 \mbox{{\ensuremath{{}^\circ}}}$ to the right of her original course, and flies for $2 \mbox{h}$ in the new direction. \ If she
maintains a constant speed of $625 mi/\mbox{h}$ how far is she from her starting point? 

\item [33.]
A fisherman leaves his home port and heads in a direction N $70 \mbox{{\ensuremath{{}^\circ}}}$ W. \ he travels $30 \mbox{mi}$ and reaches Egg Island. \ The next day he sails N
$10 \mbox{{\ensuremath{{}^\circ}}}$ E for $50 \mbox{mi}$, reaching Forrest Island. 

\item [(a)]
Find the distance between the fisherman's home port and Forrest Island. 

\item [(b)]
Find the bearing from Forrest Island back to his home port. 

\item [35.]
A triangular field has sides of lengths $22$, $36$, and $44$ yd. \ Find the largest angle. 

\item [37.]
A boy is flying two kites at the same time. \ He has $380 \mbox{ft}$ of line out to one kite and $420 \mbox{ft}$ of line out to the other. \ He estimates the angle
between the two lines is $30 \mbox{{\ensuremath{{}^\circ}}}$. \ Find the approximate distance between the two
kites. 

\item [39.]   
%TCIMACRO{\TeXButton{Start Two Columns}{\columnsep =30pt
% \begin {multicols}{2}}}%
%BeginExpansion
\columnsep =30pt
\begin {multicols}{2}
%EndExpansion
 A steep mountain is inclined $74 \mbox{{\ensuremath{{}^\circ}}}$ to the horizontal and rises $3400 \mbox{ft}$ above the surrounding plain. \ A cable car is to
be installed from a point $800 \mbox{ft}$ from the base to the top of the mountain, as shown. \ Find
the shortest length of cable needed. 

\item    
\setlength\fboxrule{0in}\setlength\fboxsep{0.2in}\fcolorbox[HTML]{000000}{FFFFFF}{\includegraphics[ width=2.4587in, height=1.5912in,]{L4SZ282J}
}
%TCIMACRO{\TeXButton{End Two Columns}{\end {multicols}}}%
%BeginExpansion
\end {multicols}
%EndExpansion
 \end{description}


\begin{description}
\item [41.] Three circles of radii $4$, $5$, and $6 \mbox{cm}$ respectively are mutually tangent. \ Find the area
enclosed between the circles. 

\item \qquad \qquad \qquad \qquad
\setlength\fboxrule{0in}\setlength\fboxsep{0.2in}\fcolorbox[HTML]{000000}{FFFFFF}{\includegraphics[ width=3.3269in, height=2.6394in,]{L4SZ282K}
}


\item [43.]   
%TCIMACRO{\TeXButton{Start Two Columns}{\columnsep =30pt
% \begin {multicols}{2}}}%
%BeginExpansion
\columnsep =30pt
\begin {multicols}{2}
%EndExpansion
 A surveyor wishes to find the distance between two points $A$ and $B$ on the opposite side of a river. \ on her side of the river she chooses two points
$C$ and $D$ that are $20 \mbox{m}$ apart and measures the angles shown. \ Find
the distance between $A$ and $B\text{.}$ 

\item    
\setlength\fboxrule{0in}\setlength\fboxsep{0.2in}\fcolorbox[HTML]{000000}{FFFFFF}{\includegraphics[ width=2.5028in, height=1.9995in,]{L4SZ282L}
}
%TCIMACRO{\TeXButton{End Two Columns}{\end {multicols}}}%
%BeginExpansion
\end {multicols}
%EndExpansion
 

\item [45.]
Land in downtown Columbia is valued at $ \$20$ a square foot. \ What is the value of a triangular lot with sides of lengths $112$, $148$, and $190 \mbox{ft}$? \end{description}


%TCIMACRO{\TeXButton{Start Two Columns}{\columnsep =30pt
% \begin {multicols}{2}}}%
%BeginExpansion
\columnsep =30pt
\begin {multicols}{2}
%EndExpansion
 


%TCIMACRO{\TeXButton{End Two Columns}{\end {multicols}}}%
%BeginExpansion
\end {multicols}
%EndExpansion
 

\section{3.6 Answers}
%TCIMACRO{\TeXButton{Start Two Columns}{\columnsep =30pt
% \begin {multicols}{2}}}%
%BeginExpansion
\columnsep =30pt
\begin {multicols}{2}
%EndExpansion
 

\textbf{Exercises 3.1} 

1. $\frac{\pi }{5} \approx 0.628 \mbox{rad}$ 

3. $\frac{ -8 \pi }{3} \approx  -8.378 \mbox{rad}$ 

5. $\frac{\pi }{3} \approx 1.047 \mbox{rad}$ 

7. $\frac{ -3 \pi }{4} \approx  -2.356 \mbox{rad}$ 

9. $135 \mbox{{\ensuremath{{}^\circ}}}$ 

11. $150 \mbox{{\ensuremath{{}^\circ}}}$ 

13. $\frac{ -270}{\pi } \approx  -85.9 \mbox{{\ensuremath{{}^\circ}}}$ 

15. $ -15 \mbox{{\ensuremath{{}^\circ}}}$ 

41. $\frac{55 \pi }{9} \approx 19.2$ 

43. $4$ 

45. $4 \mbox{mi}$ 

47. $2 \mbox{rad} \approx 114.6 \mbox{{\ensuremath{{}^\circ}}}$ 

49. $\frac{36}{\pi } \approx 11.459 \mbox{m}$ 

51. $330 \pi  \approx 1037 \mbox{mi}$ 

53. $1.6$ million $\mbox{mi}$ 

55. $1.15 \mbox{mi}$ 

57. $50 \mathrm{m}^{2}$ 

59. $4 \mbox{m}$ 

61. $6 cm^{2}$ 

63. $\frac{32 \pi }{15} ft/\mbox{s}$ 

65.(a) $2000 \pi  rad/\mbox{min}$ (b) $\frac{50 \pi }{3} ft/\mbox{s} \approx 52.4 ft/\mbox{s}$ 

67. $39.3 mi/\mbox{h}$ 

69. $2.1 \mathrm{m}/\mbox{s}$ 

\textbf{Exercises 3.2} 

1. $\sin  \theta  =\frac{4}{5} ,\cos  \theta  =\frac{3}{5} ,\tan  \theta  =\frac{4}{3}$ 

3. $\sin  \theta  =\frac{40}{41} ,\cos  \theta  =\frac{9}{41} ,\tan  \theta  =\frac{40}{9}$ 

5. $\sin  \theta  =\frac{2 \sqrt{13}}{13} ,\cos  \theta  =\frac{3 \sqrt{13}}{13} ,\tan  \theta  =\frac{2}{3}$ 

10. $12 \sqrt{2}$ 

11. $\frac{13 \sqrt{3}}{2}$ 

13. $16.51658$ 

29. $45 \mbox{{\ensuremath{{}^\circ}}} ,16 ,16 \sqrt{2} \approx 22.63$ 

31. $38 \mbox{{\ensuremath{{}^\circ}}} ,44.79 ,56.85$ 

35. $1026 \mbox{ft}$ 

37.(a) $2100 \mbox{mi}$ (b) No 

39. $19 \mbox{ft}$ 

42. $600 \sin  65 \mbox{{\ensuremath{{}^\circ}}} \approx 544 \mbox{ft}$ 

43. $345 \mbox{ft}$ 

45. $415 \mbox{ft} ,152 \mbox{ft}$ 

46.$11,379 \mbox{ft}$ 

49. $5808 \mbox{ft}$ 

\textbf{Exercises 3.3} 

7. $\frac{1}{2}$ 

9. $ -\frac{\sqrt{2}}{2}$ 

11. $ -\sqrt{3}$ 

15. $ -\frac{\sqrt{3}}{2}$ 

17. $\frac{\sqrt{3}}{3}$ 

19. $\frac{\sqrt{3}}{2}$ 

21. $ -1$ 

23. $\frac{1}{2}$ 

29. undefined 

41. $\cos  \theta  = -\frac{4}{5} ,\tan  \theta  = -\frac{3}{4}$ 

43. $\sin  \theta  = -\frac{3}{5} ,\cos  \theta  =\frac{4}{5}$ 

49. (a) $\frac{\sqrt{3}}{2} ,\sqrt{3}$ (b) $\frac{1}{2} ,\frac{\sqrt{3}}{4}$ (c) $\frac{3}{4} ,0.88967$ 

51. $19.1$ 

53. $66.1 \mbox{{\ensuremath{{}^\circ}}}$ 

\textbf{Exercises 3.4} 

1. $318.8$ 

5. $44 \mbox{{\ensuremath{{}^\circ}}}$ 

7. $\angle C =114 \mbox{{\ensuremath{{}^\circ}}} ,a \approx 51 ,b \approx 24$ 

9. $\angle C =62 \mbox{{\ensuremath{{}^\circ}}} ,a \approx 200 b \approx 242$ 
%TCIMACRO{\TeXButton{End Two Columns}{\end {multicols}}}%
%BeginExpansion
\end {multicols}
%EndExpansion
  

13. $\angle A =100 \mbox{{\ensuremath{{}^\circ}}} ,a \approx 89 ,c \approx 71$ 

15. $\angle B \approx 30 \mbox{{\ensuremath{{}^\circ}}} ,\angle C \approx 40 \mbox{{\ensuremath{{}^\circ}}} ,c \approx 19$ 

17. no solution 

19. $\angle A_{1} \approx 125 \mbox{{\ensuremath{{}^\circ}}} ,\angle C_{1} \approx 30 \mbox{{\ensuremath{{}^\circ}}} ,a_{1} \approx 49\text{,}$ 

\ \ \ \ \ \ $\angle A_{2} \approx 5 \mbox{{\ensuremath{{}^\circ}}} ,\angle C_{2} \approx 150 \mbox{{\ensuremath{{}^\circ}}} ,a_{2} \approx 5.6$ 

23. $219 \mbox{ft}$ 

25. (b) $1018 \mbox{mi}\text{,}$ (c) $1017 \mbox{mi}$ 

27. $155 \mbox{m}$ 

31. $48.2 \mbox{{\ensuremath{{}^\circ}}}$ 

\textbf{Exercises 3.5} 

1. $28.9$ 

5. $29.89 \mbox{{\ensuremath{{}^\circ}}}$ 

9. $\angle A \approx 39.4 \mbox{{\ensuremath{{}^\circ}}} ,\angle B \approx 20.6 \mbox{{\ensuremath{{}^\circ}}} ,c \approx 24.6$ 

13. $\angle A \approx 50 \mbox{{\ensuremath{{}^\circ}}} ,\angle B \approx 73 \mbox{{\ensuremath{{}^\circ}}} ,\angle C \approx 57 \mbox{{\ensuremath{{}^\circ}}}$ 

15. $\angle A_{1} \approx 83.6 \mbox{{\ensuremath{{}^\circ}}} ,\angle C_{1} \approx 56.4 \mbox{{\ensuremath{{}^\circ}}} ,a_{1} \approx 193$ 

\ \ \ \ \ \ $\angle A_{2} \approx 16.4 \mbox{{\ensuremath{{}^\circ}}} ,\angle C_{2} \approx 123.6 \mbox{{\ensuremath{{}^\circ}}} ,a_{2} \approx 54.9$ 

19. $2$ 

23. $84.6 \mbox{{\ensuremath{{}^\circ}}}$ 

27. $2.30 \mbox{mi}$ 

29. $23.1 \mbox{mi}$ 

31 $2179 \mbox{mi}$ 

33. (a) $62.6 \mbox{mi}$ (b) $S 18.2 \mbox{{\ensuremath{{}^\circ}}} E$ 

35. $96 \mbox{{\ensuremath{{}^\circ}}}$ 

37. $211 \mbox{ft}$ 

39. $3835 \mbox{ft}$ 

41. $3.85 cm^{2}$ 

43. $14.3 \mbox{m}$ 

45. $ \$165,554$ 
