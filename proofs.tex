%-----------------------------------------------------------------------
Proof of the product rule:
\textbf{Proof} 

Let $u =f (x)$ and $v =g (x)$. If $x$ changes by an amount $ \Delta x$ then the corresponding changes in $u$ and $v$ are $ \Delta u =f (x + \Delta x) -f (x)\text{\quad \quad }$and$\text{\quad \quad } \Delta v =g (x + \Delta x) -g (x)$ 

The corresponding change in the product $u v$ is 


\begin{align*} \Delta (u v) &  = \left (u + \Delta u\right ) \left (v + \Delta v\right ) -u v \\
 &  = u v +u  \Delta v +v  \Delta u + \Delta u  \Delta v -u v \\
 &  = u  \Delta v +v  \Delta u + \Delta u  \Delta v\end{align*}

If we divide by $ \Delta x$, we get
\begin{equation*}\frac{ \Delta (u v)}{ \Delta x} =u \frac{ \Delta v}{ \Delta x} +v \frac{ \Delta u}{ \Delta x} + \Delta u \frac{ \Delta v}{ \Delta x}
\end{equation*}

Now we let $ \Delta x \rightarrow 0$
\begin{align*}\underset{}{\frac{d}{d x} \left (u v\right ) =\underset{ \Delta x \rightarrow 0}{\lim }} \genfrac{[}{]}{}{}{ \Delta (u v)}{ \Delta x} &  = \underset{ \Delta x \rightarrow 0}{\lim }\left [u \frac{ \Delta v}{ \Delta x}\right ] +\underset{}{\underset{ \Delta x \rightarrow 0}{\lim }\left [v \frac{ \Delta u}{ \Delta x}\right ]} +\underset{}{\underset{ \Delta x \rightarrow 0}{\lim }\left [ \Delta u \frac{ \Delta v}{ \Delta x}\right ]} \\
\frac{d}{d x} (u v) &  = u \frac{d v}{d x} +v \frac{d u}{d x} +0. \frac{d v}{d x}\end{align*}

So
\begin{equation*}\frac{d}{d x} \left [f (x) g (x)\right ] =f (x) \frac{d}{d x} \left [g (x)\right ] +g (x) \frac{d}{d x} \left [f (x)\right ]
\end{equation*}

%-----------------------------------------------------------------------
Proof of the quotient rule:
\begin{align*} \Delta \genfrac{(}{)}{}{}{u}{v} &  = \frac{u + \Delta u}{v + \Delta v} -\frac{u}{v} =\frac{\left (u + \Delta u\right ) v -u \left (v + \Delta v\right )}{v \left (v + \Delta v\right )} \\
&  = \frac{v  \Delta u -u  \Delta v}{v \left (v + \Delta v\right )}\end{align*}

If we divide by $ \Delta x$, we get
\begin{equation*}\frac{ \Delta \genfrac{(}{)}{}{}{u}{v}}{ \Delta x} =\frac{v \frac{ \Delta u}{ \Delta x} -u \frac{ \Delta v}{ \Delta x}}{v \left (v + \Delta v\right )}
\end{equation*}

So
\begin{equation*}\frac{d}{d x} \genfrac{(}{)}{}{}{u}{v} =\underset{ \Delta x \rightarrow 0}{\lim }\frac{ \Delta \genfrac{(}{)}{}{}{u}{v}}{ \Delta x} =\underset{ \Delta x \rightarrow 0}{\lim }\frac{v \frac{ \Delta u}{ \Delta x} -u \frac{ \Delta v}{ \Delta x}}{v \left (v + \Delta v\right )}
\end{equation*}

As $ \Delta x \rightarrow 0$, $ \Delta v \rightarrow 0$, $\frac{ \Delta u}{ \Delta x} \rightarrow \frac{d u}{d x}$ and $\frac{ \Delta v}{ \Delta x} \rightarrow \frac{d v}{d x}$ 

So
\begin{equation*}\frac{d}{d x} \genfrac{(}{)}{}{}{u}{v} \rightarrow \frac{v \frac{d u}{d x} -u \frac{d v}{d x}}{v^{2}}
\end{equation*}

%-----------------------------------------------------------------------
Chain Rule
The proof is not as straightforward as the others we have met so far so will not be covered in the course. 

To illustrate the difficulty let us proceed as we have done in the proofs of the product rule and quotient rule until we find a flaw in our reasoning.


Let $u =g (x)$ then if $x$ is given an increment of $ \Delta x$ the corresponding increment in $u$ is $ \Delta u$. So
\begin{equation*} \Delta u =g (x + \Delta x) -g (x)
\end{equation*}

And if $y =f (g (x))$ then $y =f (u)$ so the corresponding change in $y$ is
\begin{equation*} \Delta y =f (u + \Delta u) -f (u)
\end{equation*}

We can write
\begin{equation*}\frac{d y}{d x} =\underset{ \Delta x \rightarrow 0}{\lim }\frac{ \Delta y}{ \Delta x}
\end{equation*}

But we are not able to expand this to become
\begin{align*} &  = \underset{ \Delta x \rightarrow 0}{\lim }\frac{ \Delta y}{ \Delta u} \cdot \frac{ \Delta u}{ \Delta x} \\
&  = \underset{ \Delta x \rightarrow 0}{\lim }\frac{ \Delta y}{ \Delta u} \cdot \underset{ \Delta x \rightarrow 0}{\lim } \frac{ \Delta u}{ \Delta x} \\
&  = \underset{ \Delta u \rightarrow 0}{\lim }\frac{ \Delta y}{ \Delta u} \cdot \underset{ \Delta x \rightarrow 0}{\lim } \frac{ \Delta u}{ \Delta x} \\
&  = \frac{d y}{d u} \cdot \frac{d u}{d x}\end{align*}

The reason we can't prove the rule in this way is that $ \Delta u$ could be equal to $0$ even when $ \Delta x \neq 0$ which would mean we are dividing by zero. However the procedure at least suggests
the chain rule is feasible. 


%-----------------------------------------------------------------------
trig functions
\subsection{The Derivative of $f (x) =\sin  x$ is $f^{ \prime } (x) =\cos  x$}
The proof from first principles requires us to use a trigonometric identity that you may not have met
\begin{equation*}\sin  (x +h) =\sin  x \cos  \text{}h +\cos  x \sin  \text{}h
\end{equation*}

This is called an addition formula and can be found in trigonometry textbooks. We
have
\begin{align*}f^{ \prime } (x) &  = \underset{h \rightarrow 0}{\lim }\frac{f (x +h) -f (x)}{h} \\
&  = \underset{h \rightarrow 0}{\lim }\frac{\sin  (x +h) -\sin  x}{h} \\
&  = \underset{h \rightarrow 0}{\lim }\frac{\sin  x \cos  \text{}h +\cos  x \sin  \text{}h -\sin  x}{h} \\
&  = \underset{h \rightarrow 0}{\lim }\left [\frac{\sin  x \cos  \text{}h -\sin  x}{h} +\frac{\cos  x \sin  \text{}h}{h}\right ] \\
&  = \underset{h \rightarrow 0}{\lim }\left [\sin  x \genfrac{(}{)}{}{}{\cos  \text{}h -1}{h} +\cos  x \genfrac{(}{)}{}{}{\sin  \text{}h}{h}\right ] \\
&  = \underset{h \rightarrow 0}{\lim }\sin  x \cdot \underset{h \rightarrow 0}{\lim } \frac{\cos  \text{}h -1}{h} +\underset{h \rightarrow 0}{\lim }\cos  x \cdot \underset{h \rightarrow 0}{\lim } \frac{\sin  \text{}h}{h}\end{align*}

We say that when we evaluate a limit as $h \rightarrow 0$ we regard $x$ as being constant so $\underset{h \rightarrow 0}{\lim }\sin  x =\sin  x$ and $\underset{h \rightarrow 0}{\lim }\cos  x =\cos  x$. It can be shown that $\underset{h \rightarrow 0}{\lim }\frac{\sin  \text{}h}{h} =1$ and $\underset{h \rightarrow 0}{\lim }\frac{\cos  \text{}h -1}{h} =0$. These two limits can be illustrated on a spreadsheet by tabulating values of $\frac{\sin  \text{}h}{h}$ and $\frac{\cos  \text{}h -1}{h}$ for values of $h$ that are made to approach zero (i.e. $h \rightarrow 0$). If these 4 results are substituted in the formula for $f^{ \prime } (x)$ above we get
\begin{align*}f^{ \prime } (x) &  = \sin  x \cdot 0 +\cos  x \cdot 1 \\
&  = \cos  x\end{align*}

Another way of expressing this result is
\begin{equation*}\frac{d}{d x} \left (\sin  x\right ) =\cos  x
\end{equation*}

\subsection{The Derivative of $g (x) =\cos  x$ is $g^{ \prime } (x) = -\sin  x$}
A method similar to that given in the previous section can be used to prove this formula from first principles. Again
it depends on us being able to use another trigonometric identity. 

You could verify this formula if you wish. You
will need to use the addition formula
\begin{equation*}\cos  (x +h) =\cos  x \cos  \text{}h -\sin  x \sin  \text{}h
\end{equation*}

together with the limit rules given above. 

\subsection{The Derivative of $h (x) =\tan  x$ is $h^{ \prime } (x) =\sec ^{2} x$}
This formula is proved using the trigonometric identity $\tan  x =\frac{\sin  x}{\cos  x}$ and the \emph{Quotient Rule}.